\documentclass[11pt, twoside, a4paper]{article}

% Setup
\usepackage[margin=2.4cm, top=3.5cm]{geometry}
\usepackage[utf8]{inputenc}
\usepackage[ngerman]{babel}

% Package imports
\usepackage{amsfonts}
\usepackage{amsmath}
\usepackage{amssymb}
\usepackage{amsthm}
\usepackage{mathtools}
\usepackage{setspace}
\usepackage{float}
\usepackage{enumitem}
\usepackage{hyperref}
\usepackage[pagestyles]{titlesec}
\usepackage{fancyhdr}
\usepackage{colonequals}
\usepackage{caption}
\usepackage{tikz}
\usepackage{marginnote}
\usepackage{etoolbox}
\usepackage{mdframed}
\usepackage{aligned-overset}
\usepackage{esint}

% Font-Encoding
\usepackage[T1]{fontenc}
\usepackage{lmodern}

% Theorems
\newtheoremstyle{plain}{}{}{}{}{\bfseries}{.}{ }{}
\theoremstyle{plain}
\newtheorem{blockelement}{Blockelement}[subsection]
\newtheorem{bemerkung}[blockelement]{Bemerkung}
\newtheorem{definition}[blockelement]{Definition}
\newtheorem{lemma}[blockelement]{Lemma}
\newtheorem{satz}[blockelement]{Satz}
\newtheorem{notation}[blockelement]{Notation}
\newtheorem{korollar}[blockelement]{Korollar}
\newtheorem{uebung}[blockelement]{Übung}
\newtheorem{beispiel}[blockelement]{Beispiel}
\newtheorem{folgerung}[blockelement]{Folgerung}
\newtheorem{axiom}[blockelement]{Axiom}
\newtheorem{beobachtung}[blockelement]{Beobachtung}
\newtheorem{konzept}[blockelement]{Konzept}
\newtheorem{visualisierung}[blockelement]{Visualisierung}
\newtheorem{anwendung}[blockelement]{Anwendung}
\newtheorem{skizze}[blockelement]{Skizze}
\newtheorem{genv}[blockelement]{}

% Equation numbering
\numberwithin{equation}{subsection}
\newcommand{\numberthis}[0]{\addtocounter{equation}{1}\tag{\theequation}}

% Marginnotes left
\makeatletter
\patchcmd{\@mn@@@marginnote}{\begingroup}{\begingroup\@twosidefalse}{}{\fail}
\reversemarginpar
\makeatother

% Long equations
\allowdisplaybreaks

% \left \right
\newcommand{\set}[1]{\left\{#1\right\}}
\newcommand{\pair}[1]{\left(#1\right)}
\newcommand{\of}[1]{\mathopen{}\mathclose{}\bgroup\left(#1\aftergroup\egroup\right)}
\newcommand{\abs}[1]{\left\lvert#1\right\rvert}
\newcommand{\norm}[1]{\left\lVert#1\right\rVert}
\newcommand{\linterv}[1]{\left[#1\right)}
\newcommand{\rinterv}[1]{\left(#1\right]}
\newcommand{\interv}[1]{\left[#1\right]}
\newcommand{\sprod}[1]{\left<#1\right>}

% Shorten commands
\newcommand{\equivalent}[0]{\Leftrightarrow{}}
\newcommand{\impl}[0]{\Rightarrow{}}
\newcommand{\fromto}{\rightarrow{}}
\newcommand{\definedas}[0]{\coloneqq}
\newcommand{\definedasbackwards}[0]{\eqqcolon}
\newcommand{\definedasequiv}[0]{\ratio\Leftrightarrow{}}
\newcommand{\exclude}[0]{\setminus}
\renewcommand{\emptyset}{\varnothing}
\newcommand{\sbset}{\subseteq}
\newcommand{\dif}{\mathop{}\!\mathrm{d}}

\newcommand{\ntoinf}[0]{n\fromto\infty}
\newcommand{\toinf}{\fromto\infty}
\newcommand{\fa}{\;\forall\,}
\newcommand{\ex}{\;\exists\,}
\newcommand{\conj}[1]{\overline{#1}}

\newcommand{\annot}[3][]{\overset{\text{#3}}#1{#2}}
\newcommand{\biglim}[1]{{\displaystyle \lim_{#1}}}
\newcommand{\nn}[0]{\\[2\baselineskip]}
\newcommand{\anf}[1]{\glqq{}#1\grqq}
\newcommand{\OBDA}{o.B.d.A. }
\newcommand{\theoremescape}{\leavevmode}
\newcommand{\aligntoright}[2]{\hfill#1\hspace{#2\textwidth}~}
\newcommand{\horizontalline}[0]{\par\noindent\rule{0.05\textwidth}{0.1pt}\\}
\newcommand{\rgbcolor}[3]{rgb,255:red,#1;green,#2;blue,#3}
\newcommand{\fixedspace}[2]{\makebox[#1][l]{#2}}
\newcommand{\ov}[1]{\overline{#1}}
\newcommand{\un}[1]{\underline{#1}}
\newcommand{\verteq}{\rotatebox{-90}{$~=$}}
\newcommand{\equalto}[2]{\underset{\scriptstyle\overset{\mkern4mu\verteq}}{#1}}
\newcommand{\eqbelow}[1]{\underset{\verteq}{#1}}

\let\Re\relax
\let\Im\relax

% MathOperators
\DeclareMathOperator{\grad}{Grad}
\DeclareMathOperator{\bild}{Bild}
\DeclareMathOperator{\Re}{Re}
\DeclareMathOperator{\Im}{Im}
\DeclareMathOperator{\arcsinh}{arcsinh}
\DeclareMathOperator{\arccosh}{arccosh}


% Mengenbezeichner
\newcommand{\R}{\mathbb{R}}
\newcommand{\N}{\mathbb{N}}
\newcommand{\C}{\mathbb{C}}
\newcommand{\Z}{\mathbb{Z}}
\newcommand{\Q}{\mathbb{Q}}
\newcommand{\K}{\mathbb{K}}

\newcommand{\mR}{\mathcal{R}}
\newcommand{\mB}{\mathcal{B}}
\newcommand{\mC}{\mathcal{C}}
\newcommand{\mJ}{\mathcal{J}}
\newcommand{\mPC}{\mathcal{PC}}

\newcommand\imaginarysubsection[1]{
    \refstepcounter{subsection}
    \subsectionmark{#1}
}

% Unfassbar hässlich, aber effektiv für temporäre schnelle Lösungen
\def\:={\coloneqq}
\def\->{\fromto}
\def\=>{\impl}
\def\<={\leq}
\def\>={\geq}

% Envs
\newenvironment{induktionsanfang}{
    \rule{0pt}{3ex}\noindent
    \begin{minipage}[t]{0.11\textwidth}
    {I-Anfang}
    \end{minipage}
    \hfill
    \begin{minipage}[t]{0.89\textwidth}
    }
    {
    \end{minipage}
}
\newenvironment{induktionsvoraussetzung}{
    \rule{0pt}{3ex}\noindent
    \begin{minipage}[t]{0.11\textwidth}
    {I-Vor.}
    \end{minipage}
    \hfill
    \begin{minipage}[t]{0.89\textwidth}
    }
    {
    \end{minipage}
}
\newenvironment{induktionsschritt}{
    \rule{0pt}{3ex}\noindent
    \begin{minipage}[t]{0.11\textwidth}
    {I-Schritt}
    \end{minipage}
    \hfill
    \begin{minipage}[t]{0.89\textwidth}
    }
    {
    \end{minipage}
}

% Section style
\titleformat*{\section}{\LARGE\bfseries}
\titleformat*{\subsection}{\large\bfseries}

% Page styles
\newpagestyle{pagenumberonly}{
    \sethead{}{}{}
    \setfoot[][][\thepage]{\thepage}{}{}
}
\newpagestyle{headfootdefault}{
    \sethead[][][\thesubsection~\textit{\subsectiontitle}]{\thesection~\textit{\sectiontitle}}{}{}
    \setfoot[][][\thepage]{\thepage}{}{}
}
\pagestyle{headfootdefault}

\begin{document}
    \title{\vspace{3cm} Skript zur Vorlesung\\Analysis II\\bei Prof. Dr. Dirk Hundertmark}
    \author{Karlsruher Institut für Technologie}
    \date{Sommersemester 2024}
    \maketitle
    \begin{center}
        Dieses Skript ist inoffiziell. Es besteht kein\\Anspruch auf Vollständigkeit oder Korrektheit.
    \end{center}
    \thispagestyle{empty}
    \newpage

    \tableofcontents
    ~\\
    Alle mit [*] markierten Kapitel sind noch nicht Korrektur gelesen und bedürfen eventuell noch Änderungen.

    \newpage

    \include{Kapitel/Riemann_Integral}
    \include{Kapitel/Orientiertes_Riemann_Integral}
    \include{Kapitel/Hauptsatz_Integral_Differentialrechnung}
    \section{[*] Uneigentliche Integrale}
\thispagestyle{pagenumberonly}
Bisher haben wir immer nur Integrale auf kompakten Intervalle $I$ berechnet und dabei waren alle Funktionen $f\in\mR\of{I}$ insbesondere beschränkt.\\
Frage: Was ist $ \int_{0}^{1} \frac{1}{\sqrt{x}} \dif x$? Was ist $ \int_{0}^{\infty} e^{-t} \dif t$?
\begin{align*}
    \int_{a}^{b} e^{-t} \dif t &= \interv{-e^{-t}}_a^b = e^{-0} - e^{-b} = 1 - e^{-b} = 1 -\frac{1}{e^b}\fromto 1 \text{ für } b\fromto\infty
\end{align*}

\subsection{Uneigentliche Integrale: Fall I}
Es sei $I=\linterv{a, \infty}$, $f: I\fromto\R$ und $f\in\mR\of{\interv{a,b}}~\forall a<b<\infty$ sowie $F\of{b} = \int_{a}^{b} f\of{x} \dif x$.
\begin{definition}[Fall]
    Wir definieren
    \begin{align*}
        \int_{a}^{\infty} f\of{x} \dif x &\coloneqq \lim_{b\toinf} F\of{b} = \lim_{b\toinf} \int_{0}^{b} f\of{x} \dif x
    \end{align*}
    sofern der Grenzwert existiert nennen wir das das uneigentliche Integral von $f$ über $\linterv{a,\infty}$. Wenn der Grenzwert existiert, sagen wir das Integral konvergiert.\\
    Divergiert das Integral und gilt $F\of{b}\toinf$ für $b\toinf$ (oder $F\of{b}\fromto -\infty$ für $b\toinf$), so nennen wir das Integral bestimmt divergent und schreiben
    \begin{align*}
        \int_{a}^{\infty} f\of{x} \dif x &= +\infty
        \intertext{oder}
        \int_{a}^{\infty} f\of{x} \dif x &= -\infty
    \end{align*}
\end{definition}

\begin{satz} % Satz 2
    \label{satz:int-uneigentlich-epsilon}
    Das Integral $ \int_{a}^{\infty} f\of{x} \dif x$ existiert genau dann, wenn
    \begin{align*}
        \forall\varepsilon > 0\ex R\geq a\colon \abs{F\of{b_2} - F\of{b_1}} &= \abs{ \int_{b_1}^{b_2} f\of{x} \dif x} < \varepsilon\quad\forall b_1, b_2 \geq R
    \end{align*}
    \begin{proof}
        Wir wollen die Existenz von $\biglim{b\toinf} F\of{b}$ für $F\of{b} = \int_{a}^{b} f\of{x} \dif x$. Dann folgt der Satz aus dem Cauchy-Kriterium für Grenzwerte.
    \end{proof}
\end{satz}

\begin{definition}[Absolut konvergente uneigentliche Integrale]
    Das Integral
    \begin{align*}
        \int_{a}^{\infty} f\of{x} \dif x
        \intertext{heißt absolut konvergent, falls}
        \int_{a}^{\infty} \abs{f\of{x}} \dif x
    \end{align*}
    konvergiert.
\end{definition}

\begin{satz}
    Ist das Integral $\int_{a}^{\infty} f\of{x} \dif x$ absolut konvergent, so ist es auch konvergent. Das heißt ist $ \int_{a}^{\infty} \abs{f\of{x}} \dif x < \infty$, so konvergiert auch $ \int_{a}^{\infty} f\of{x} \dif x$.
    \begin{proof}
        Wir setzen $G\of{b} = \int_{a}^{b} \abs{f\of{x}} \dif x$ und $F\of{b} = \int_{a}^{b} f\of{x} \dif x$. Wir nehmen an, dass $\biglim{b\toinf} G\of{b}$ existiert, das heißt
        \begin{align*}
            \forall\varepsilon > 0\ex R\geq a\colon \abs{G\of{b_2} - G\of{b_1}} &< \varepsilon\quad\forall b_1, b_2\geq R\\
            \impl \abs{F\of{b_2} - F\of{b_1}} &= \abs{ \int_{b_1}^{b_2} f\of{x} \dif x}\\
            &\leq \int_{b_1}^{b_2} \abs{f\of{x}} \dif x = G\of{b_2} - G\of{b_1}
        \end{align*}
        Damit folgt die Behauptung aus Satz~\ref{satz:int-uneigentlich-epsilon}.
    \end{proof}
\end{satz}

\begin{satz} % Satz 5
    \label{satz:int-majorant}
    Sei $\varphi: \linterv{a,\infty}\fromto\linterv{0, \infty}$ mit
    \begin{align*}
        \int_{a}^{\infty} \varphi\of{x} \dif x &< \infty
        \intertext{une es existiert ein $R_0 \geq 0$, sodass}
        \abs{f\of{x}} &\leq \varphi\of{x}\quad\forall x \geq R_0
        \intertext{Dann ist}
        \int_{a}^{\infty} f\of{x} \dif x
    \end{align*}
    absolut konvergent.
    \begin{proof}
        Für $b_2 \geq b_1\geq R_0$ gilt
        \begin{align*}
            \abs{F\of{b_2} - F\of{b_1}} &= \abs{ \int_{b_1}^{b_2} f\of{x} \dif x}\\
            &\leq \int_{b_1}^{b_2} \abs{f\of{x}} \dif x < \int_{b_1}^{b_2} \varphi\of{x} \dif x\\
            &\leq \int_{b_1}^{b_2} \varphi\of{x} \dif x\fromto 0 \text{ für } b_1\toinf
        \end{align*}
    \end{proof}
\end{satz}

\begin{beispiel}
    Das Integral
    \begin{align*}
        \int_{a}^{\infty} \frac{\sin x}{x} \dif x&
        \intertext{ist konvergent, aber nicht absolut konvergent. Wir definieren}
        f\of{x} &= \begin{cases}
                       \frac{\sin x}{x} &x\neq 0\\
                       1 &x = 0
        \end{cases}
        \intertext{Damit ist $f$ stetig auf $\pair{-\infty, \infty}$ und damit folgt $f\in\mR\of{\interv{a,b}}~\forall a,b\in\R$. Insbesondere existiert}
        \int_{0}^{1} \frac{\sin x}{x} \dif x&\\
        \int_{a}^{b} \frac{\sin x}{x} \dif x &= \int_{a}^{1} \frac{\sin x}{x} \dif x + \int_{1}^{b} \frac{\sin x}{x} \dif x\\
        \int_{1}^{b} \frac{\sin x}{x} \dif x &= \interv{-\cos + \frac{1}{x}}_1^b - \int_{1}^{b} \frac{\cos x}{x^2} \dif x\\
        &= \cos 1 - \frac{\cos b}{b} - \int_{1}^{b} \frac{\cos x}{x^2} \dif x
        \intertext{Wir definieren $\varphi\of{x} = \frac{1}{x^2}$ mit}
        \int_{1}^{b} \frac{1}{x^2} \dif x &= \interv{-\frac{1}{x}}_1^b = 1 - \frac{1}{b}\fromto 1
        \intertext{Außerdem gilt}
        \abs{\frac{\cos x}{x^2}} &\leq \frac{1}{x^2}
        \intertext{Damit ist das Integral nach dem Majorantenkriterium konvergent. Um einzusehen, dass es nicht absolut konvergent ist, betrachten wir für $N\in\N$}
        \int_{N\pi}^{\pair{N+1}\pi} \abs{\frac{\sin x}{\pi}} \dif x &= \int_{N\pi}^{\abs{N+1}\pi} \frac{\abs{\sin x}}{x} \dif x\\
        &\geq \frac{1}{\pi\pair{N+1}} \cdot \int_{N\pi}^{\pair{N+1}\pi} \abs{\sin x} \dif x\\
        \impl \int_{0}^{\pair{k+1}\pi} \abs{\frac{\sin x}{x}} \dif x &= \sum_{n=0}^{k} \int_{n\pi}^{\pair{n+1}\pi} \frac{\abs{\sin x}}{x} \dif x\\
        &\geq \sum_{n=0}^{k} \frac{2}{\pi\pair{n+1}} = \frac{2}{\pi} \sum_{n=0}^{k} \frac{1}{n+1}\fromto \infty
    \end{align*}
\end{beispiel}

\begin{bemerkung}
    Analog zu $\linterv{a, \infty}$ wollen wir auch die Integrale in $\rinterv{-\infty, b}$ betrachten. Wir setzen
    \begin{align*}
        F\of{a} &= \int_{a}^{b} f\of{x} \dif x\\
        \int_{-\infty}^{b} f\of{x} \dif x &\coloneqq \lim_{a\fromto -\infty} \int_{a}^{b} f\of{x} \dif x
    \end{align*}
    sofern der Grenzwert existiert. Alle Aussagen für $\linterv{a, \infty}$ gelten analog auch für $\rinterv{-\infty, b}$.
\end{bemerkung}

\begin{definition} % Definition 6
    Sei $f: \pair{-\infty, \infty}\fromto \R$ und $f\in\mR\of{\interv{a,b}}~\forall a,b\in\R$. Dann nehmen wir $c\in\R$ beliebig und definieren, dass
    \begin{align*}
        \int_{-\infty}^{\infty} f\of{x} \dif x&
        \intertext{konvergiert, falls}
        \int_{-\infty}^{c} f\of{x} \dif& \text{ und } \int_{c}^{\infty} f\of{x} \dif x
        \intertext{beide konvergieren. Und setzen}
        \int_{-\infty}^{\infty} f\of{x} \dif x &\coloneqq \int_{-\infty}^{c} f\of{x} \dif x + \int_{c}^{\infty} f\of{x} \dif x
    \end{align*}
\end{definition}

\begin{uebung}
    Weisen Sie nach, dass sowohl die Konvergenz, als auch der Wert des Integrals in der vorherigen Definition unabhängig von der Wahl von $c$ ist.
\end{uebung}

\begin{bemerkung}
    Es ist allerdings zu beachten, dass
    \begin{align*}
        \lim_{a\toinf} \int_{a}^{c} f\of{x} \dif x + \lim_{b\toinf} \int_{c}^{b}  \dif x &\neq \lim_{R\toinf} \int_{-R}^{R} f\of{x} \dif x
    \end{align*}
    Das heißt die Integrale müssen tatsächlich getrennt betrachtet werden. Zum Beispiel bei der Funktion $f\of{x} = x$ geht $ \int_{-R}^{R} x \dif x \fromto 0$, aber ist eigentlich nicht auf $\pair{-\infty, \infty}$ integrierbar, da sich bei der Trennung in zwei Integrale kein Grenzwert ergibt.
\end{bemerkung}

\subsection{Uneigentliche Integrale: Fall II}
Es sei $I=\linterv{a, b}$ (oder $I=\rinterv{a, b}$) und $f:I\fromto\R$ unbeschränkt bei $x=a$ (oder $x=b$). Außerdem $f\in\mR\of{\interv{a,c}}~\forall a<c < b$ (oder $f\in\mR\of{\interv{c, b}}~\forall a < c < b$)

\begin{definition}
    Existiert
    \begin{align*}
        \lim_{c\fromto b-} \int_{a}^{c} f\of{x} \dif x\quad &\pair{\text{oder } \lim_{c\fromto a+} \int_{c}^{b} f\of{x} \dif x}
        \intertext{so setzen wir}
        \int_{a}^{b} f\of{x} \dif x = \lim_{c\fromto b-} \int_{a}^{c} f\of{x} \dif x\quad &\pair{\text{oder } \int_{a}^{b} f\of{x} \dif x = \lim_{c\fromto a+} \int_{c}^{b} f\of{x} \dif x}
        \intertext{und sagen}
        \int_{a}^{b} f\of{x} \dif x
    \end{align*}
    konvergiert.
\end{definition}

\begin{satz}
    Ist $\abs{f\of{x}} \leq \varphi\of{x}~\forall x\in\linterv{a,b}$ (oder $\forall x\in\rinterv{a,b}$) und konvergiert $ \int_{a}^{b} \varphi\of{x} \dif x$, so konvergiert auch $ \int_{a}^{b} f\of{x} \dif x$
\end{satz}

\begin{beispiel}
    Sei $f: \rinterv{0, 1}\fromto\R,~x\mapsto \frac{1}{\sqrt{x}}$. Dann gilt $F\of{x}  = 2\sqrt{x}$
    \begin{align*}
        \int_{0}^{1} \frac{1}{\sqrt{x}} \dif x &= \interv{2\sqrt{x}}_c^1 = 2-2\sqrt{c}\fromto 2
    \end{align*}
\end{beispiel}

\subsection{Uneigentliche Integrale Fall III}
\marginnote{[14. Mai]}
$f$ hat eine Singularität in $\xi$ im Inneren von $\interv{a,b}$.
\begin{beispiel}
    $f\of{x} = \frac{1}{\abs{\sqrt{x}}}$ auf $\linterv{-1, 0} \cup \rinterv{0, 1}$.
\end{beispiel}

\begin{definition}
    Wir sagen, dass
    \begin{align*}
        \int_{a}^{b} f\of{x} \dif x
        \intertext{existiert/konvergiert, falls die uneigentlichen Integrale}
        \int_{\xi}^{b} f\of{x} \dif x &\text{ und } \int_{a}^{\xi} f\of{x} \dif x
        \intertext{konvergieren. Wir setzen}
        \int_{a}^{b} f\of{x} \dif x &\coloneqq \int_{a}^{\xi} f\of{x} \dif x + \int_{\xi}^{b} f\of{x} \dif x\numberthis\label{eq:un-iii}
    \end{align*}
\end{definition}

\begin{bemerkung}
(\ref{eq:un-iii})
    ist stärker als die Existenz von
    \begin{align*}
        \lim_{\varepsilon\searrow} \int_{I_{\varepsilon}}^{} f\of{x} \dif x
    \end{align*}
    mit $I=\interv{a,b}$ und $I_{\varepsilon}\coloneqq I\exclude\pair{\xi-\varepsilon, \xi+\varepsilon} = \interv{a, \xi-\varepsilon} \cup \interv{\xi+\varepsilon, b}$. (Cauchyscher Hauptwert).
\end{bemerkung}

\begin{beispiel}
    Sei $f\of{x} = \frac{1}{x^2}$, $I=\interv{-1, 1}$. Dann existiert der Cauchysche Hauptwert, aber nicht (\ref{eq:un-iii}).
\end{beispiel}

\subsection{Uneigentliche Integrale Fall IV}

\begin{definition}
    Man hat Singularitäten in $\R$ für $f$ oder/und $b=+\infty$, $a=-\infty$. Dann zerlege $\linterv{a, \infty}$ oder $\rinterv{-\infty, b}$ oder $\pair{-\infty, \infty}$ in endlich viele Intervalle, wobei die Singularitäten die Randpunkte sind (oder $-\infty$, $\infty$). Dann existiert das Integral, falls die endlich vielen uneigentlichen Integrale existieren. Dann nehme Summe aller dieser uneigentlichen Integrale
\end{definition}

\begin{satz}[Integralvergleichskriterium] % Satz 10
    \label{satz:integral-vergleich}
    Sei $f: \linterv{1, \infty} \fromto\R$ monoton fallend. Dann gilt
    \begin{align*}
        \sum_{n=1}^{\infty}  f\of{n} \text{ konvergiert } \equivalent \int_{1}^{\infty} f\of{x} \dif x \text{ existiert }
    \end{align*}
    \begin{proof}
        Siehe Saalübung.
    \end{proof}
\end{satz}

\begin{beispiel}
    Es sei $f\of{x} = x^{-p}$ mit $p\neq 1$. Dann ist $F\of{x} = \frac{1}{1-p}x^{1-p}$ für $F'=f$.
    \begin{align*}
        \int_{1}^{\infty} \frac{1}{x^p} \dif x &= \lim_{R\toinf} \interv{\frac{1}{1-p}x^{1-p}}_1^R
    \end{align*}
    existiert nach Satz~\ref{satz:integral-vergleich} für $p>1$.
\end{beispiel}

\begin{beispiel}
    $f\of{x} = \log_2\of{x} = \log\of{\log\of{x}}$, $x > 1$
    \begin{align*}
        \frac{\dif}{\dif x} \log_2\of{x} &= \frac{1}{\log\of{x}}\cdot\frac{1}{x}\\
        \frac{\dif}{\dif x}\pair{\log_2\of{x}}^{1-s} &= \frac{1-s}{\pair{\log x}^s}\cdot \frac{1}{x}\\
        \impl \sum_{n=2}^{\infty} \frac{1}{n\pair{\log^s n}^s} \text{ konvergiert } &\equivalent s > 1
    \end{align*}
\end{beispiel}

\begin{beispiel}[Gamma-Funktion]
    \begin{align*}
        \Gamma\of{x} &\coloneqq \int_{0}^{\infty} t^{x-1} e^{-t} \dif t\tag{$x > 0$}
    \end{align*}
    \begin{enumerate}[label=(\alph*)]
        \item
        \begin{align*}
            t^{x-1}e^{-t} &\leq t^{x-1}\quad\forall t > 0
            \intertext{\item~}
            t^{x-1}e^{-t} &= t^{x-1}e^{-\frac{t}{2}}e^{-\frac{t}{2}}\\
            &\leq c_x e^{-\frac{t}{2}}\quad\forall t\geq 1\tag{$c_x\coloneqq \sup_{t\geq 1} t^{x-1}e^{-\frac{t}{2}}$}
            \intertext{$t^{x-1}e^{-\frac{t}{2}}$ ist beschränkt auf $\linterv{1, \infty}$}
            \int_{0}^{1} t^{x-1}e^{-t} \dif t &\leq \int_{0}^{1} t^{x-1} \dif t\\
            &= \lim_{c\toinf} \interv{\frac{1}{x}t^{x}}_c^1\\
            &= \lim_{c\fromto 0^{+}} \frac{1}{x}\pair{1-e^{x}}\\
            0 &\leq \int_{1}^{\infty} t^{x-1}e^{-t} \dif t\\
            &= \lim_{b\toinf} \int_{a}^{b} t^{x-1}e^{-t} \dif t\\
            &\leq c_x e^{-\frac{t}{2}}\\
            &\leq \lim_{b\toinf} c_x \int_{a}^{b} e^{-\frac{t}{2}} \dif t < \infty\\
            \int_{a}^{b} e^{-\frac{t}{2}} &= \interv{-2e^{-\frac{t}{2}}}_1^b = 2\pair{e^{-\frac{1}{2}} - e^{-\frac{b}{2}}}\fromto 2e^{-\frac{1}{2}}
        \end{align*}
    \end{enumerate}
\end{beispiel}

\begin{satz}[Funktionalgleichung der $\Gamma$-Funktion] % Satz 12
    Es gilt $\Gamma\of{n+1} = n!$ und $x\Gamma\of{x} = \Gamma\of{x+1}$ für alle $x>0$.
    \begin{proof}
        \begin{align*}
            \Gamma\of{x+1} &= \int_{0}^{\infty} t^{(x+1)-1}e^{-t} \dif t\\
            &= \int_{0}^{\infty} t^{x}e^{-t} \dif t\\
            \intertext{Wir integrieren partiell. Sei $0 < a < b < \infty$}
            \int_{a}^{b} t^{x}e^{-t} \dif t &= \interv{-t^x e^{-t}}_a^b + \int_{a}^{b} xt^{x-1}e^{-t} \dif t\\
            &= a^{x}e^{-b} - b^{x}e^{-b} + x \int_{a}^{b} t^{x-1}e^{-t} \dif t\\
            \impl \int_{a}^{\infty} t^x e^{-t} \dif t &= \lim_{b\toinf} \int_{a}^{b} t^x e^{-t} \dif t\\
            &= a^x e^{-a} + x \int_{a}^{\infty} t^{x-1} e^{-t} \dif t\\
            \impl \Gamma\of{x+1} &= \int_{0}^{\infty} t^x e^{-t} \dif x = x\Gamma\of{x}
            \intertext{Damit folgt die zweite Behauptung. Wir betrachten außerdem}
            \Gamma\of{n+1} &= n\Gamma\of{n} = n\Gamma\of{n-1+1}\\
            &= n\pair{n-1}\Gamma\of{n-1} = n\cdot \pair{n-1}\cdot\ldots\cdot 2 \cdot 1 \cdot \Gamma\of{1}\\
            &= n!\qedhere
        \end{align*}
    \end{proof}
\end{satz}

\begin{anwendung}
    Nach Substitution mit $t^2 = x$ gilt $\frac{\dif t}{\dif x} = \frac{1}{2\sqrt{x}}$
    \begin{align*}
        \int_{a}^{\xi} e^{-t^2} \dif t &= \int_{}^{} e^{-x}\frac{1}{2}\sqrt{x} \dif x\\
        &= \frac{1}{2} \int_{0}^{\infty} \frac{1}{\sqrt{x}}e^{-x} \dif x\\
        &= \frac{1}{2} \int_{?}^{b} s^{-\frac{1}{2}} e^{-s} \dif s\\
        \intertext{für $b\toinf$ und $a\searrow 0$}
        \impl 2 \int_{0}^{\infty} e^{-t^2} \dif x &= \int_{0}^{\infty} s^{-\frac{1}{2}} e^{-s} \dif s\\
        &= \Gamma\of{\frac{1}{2}}
    \end{align*}
    Berechnung von $\Gamma\of{\frac{1}{2}}$ später.
\end{anwendung}

\newpage



    \section{[*] Integrale und gleichmäßige Konvergenz}
    \imaginarysubsection{Gleichmäßige Konvergenz}
    \thispagestyle{pagenumberonly}

    Sei $I=\interv{a,b}$ und $f: I\fromto\R$, $f_n: I\fromto \R$. Wenn die Funktionenfolge $(f_n)_n$ \anf{irgendwie} gegen $f$ konvergiert. Wann gilt dann
    \begin{align*}
        \int_{a}^{b} f_n\of{x} \dif x \fromto \int_{a}^{b} f\of{x} \dif x \text{ für } \ntoinf \text{ ?}
    \end{align*}
    Wir werden in diesem Kapitel einsehen, dass punktweise Konvergenz dafür nicht ausreichend ist, sondern wir gleichmäßige Konvergenz fordern müssen.
    \begin{beispiel}[Punktweise Konvergenz]
        Sei $f_n: \interv{0, 1}\fromto\R$ mit
        \begin{align*}
            f_n\of{x} &\coloneqq \begin{cases}
                                     n &0 < x < \frac{1}{n}\\
                                     0 &\text{sonst}
            \end{cases}
            \intertext{$(f_n)_n$ konvergiert punktweise gegen die Nullfunktion ($f_n\of{x}\fromto 0$ für $n\toinf~\forall x\in\interv{0,1}$). Außerdem gilt für ein $n\in\N$}
            \int_{0}^{1} f_n\of{x} \dif x &= \int_{0}^{\frac{1}{n}} n \dif x = \frac{n}{n} = 1
        \end{align*}
        Das Integral über die Nullfunktion ist aber 0. Das heißt punktweise Konvergenz ist kein ausreichendes Kriterium, damit die Integrale gleich sind.
    \end{beispiel}

    \begin{satz} % Satz 1
        \label{satz:gleichm-int}
        Seien $f, f_n: \interv{a,b}\fromto\R$ (oder $\C, \dots$) und $n\in\N$. Außerdem konvergiere $(f_n)_n$ gleichmäßig gegen $f$ auf $\interv{a,b}$ und $f_n\in\mR\of{\interv{a,b}}$. Dann gilt $f\in\mR\of{I}$ und
        \begin{align*}
            \lim_{\ntoinf} \int_{a}^{b} f_n\of{x} \dif x &= \int_{a}^{b} f\of{x} \dif x = \int_{a}^{b} \lim_{\ntoinf} f_n\of{x} \dif x
        \end{align*}

        \begin{proof}
            Sei $\varepsilon > 0$ und $N\in\N$ groß genug. Dann gilt
            \begin{align*}
                \norm{f-f_n}_{\infty} &= \sup_{a \leq x \leq b} \abs{f\of{x} - f_n\of{x}} < \frac{\varepsilon}{4\cdot\pair{b-a}}\\
                \impl f_n\of{x} - \frac{\varepsilon}{4\cdot\pair{b-a}} &\leq f\of{x} \leq f_n\of{x} + \frac{\varepsilon}{4\cdot\pair{b-a}}\quad\forall n\geq N\tag{1}
                \intertext{Halte $N$ fest und nehme Zerlegung $Z$ von $I=\interv{a,b}$ mit $\ov{S}_Z\of{f_N} - \un{S}_Z\of{f_N} < \frac{\varepsilon}{2}$. Dann gilt jeweils nach (1)}
                \ov{S}_Z\of{f} \leq \ov{S}_Z\of{f_N + \frac{\varepsilon}{4\cdot\pair{b-a}}} &= \ov{S}_Z\of{f_N} + \ov{S}_Z\of{\frac{\varepsilon}{4\cdot\pair{b-a}}} = \ov{S}_Z\of{f_N} + \frac{\varepsilon}{4}\\
                \un{S}_Z\of{f} \geq \un{S}_Z\of{f_N - \frac{\varepsilon}{4\cdot\pair{b-a}}} &= \un{S}_Z\of{f_N} - \un{S}_Z\of{\frac{\varepsilon}{4\cdot\pair{b-a}}} = \un{S}_Z\of{f_N} - \frac{\varepsilon}{4}\\[.6\baselineskip]
                \impl \ov{S}_Z\of{f} - \un{S}_Z\of{f} &\leq \ov{S}_Z\of{f_N} + \frac{\varepsilon}{4} - \pair{\un{S}_Z\of{f_N} - \frac{\varepsilon}{4}}\\
                &= \ov{S}_Z\of{f_N} - \un{S}_Z\of{f_N} + \frac{\varepsilon}{2}\\
                &< \frac{\varepsilon}{2} + \frac{\varepsilon}{2} = \varepsilon
                \intertext{Damit folgt $f\in\mR\of{I}$. Wir beweisen die Gleichheit der Integrale.}\\
                \int_{a}^{b} f_n\of{x} \dif x - \frac{\varepsilon}{4} &= \int_{a}^{b} \pair{f_n\of{x} - \frac{\varepsilon}{4\cdot\pair{b-a}}} \dif x\\
                &\leq \int_{a}^{b} f\of{x} \dif x \leq \int_{a}^{b} \pair{f_n\of{x} + \frac{\varepsilon}{4\cdot\pair{b-a}}} \dif x\\
                &= \int_{a}^{b} f_n\of{x} \dif x + \frac{\varepsilon}{4}\quad\forall n\geq N\\
                \impl \limsup_{\ntoinf} \int_{a}^{b} f_n\of{x} \dif x - \frac{\varepsilon}{4} &\leq \int_{a}^{b} f\of{x} \dif x \leq \liminf_{\ntoinf} \int_{a}^{b} f_n\of{x} \dif x + \frac{\varepsilon}{4}\quad\forall\varepsilon > 0\\
                \impl \limsup_{\ntoinf} \int_{a}^{b} f_n\of{x} \dif x &\leq \int_{a}^{b} f\of{x} \dif x \leq \liminf \int_{a}^{b} f_n\of{x} \dif x\qedhere
            \end{align*}
        \end{proof}
    \end{satz}

    \begin{beispiel}[Integral von Potenzreihen]
        \marginnote{[17. Mai]}
        Wir betrachten die Potenzreihe
        \begin{align*}
            f\of{x} &= \sum_{n=0}^{\infty} a_n\pair{x-x_0}^n
            \intertext{mit Konvergenzradius $R>0$ und}
            R &= \frac{1}{\displaystyle \limsup_{\ntoinf} \abs{a_n}^{\frac{1}{n}}}
            \intertext{Wir erhalten also eine Funktion $f: \pair{x_0 - R, x_0 + R}\fromto\R$ (oder $\C$). Die Stammfunktion zu $a_n\pair{x-x_0}^n$ ist $\frac{a_n}{n+1}\pair{x-x_0}^{n+1}$. Wir definieren also eine Funktion $F$ analog}
            F\of{x} = \sum_{n=0}^{\infty} \frac{a_n}{n+1}\pair{x-x_0}^{n+1} &= \sum_{n=1}^{\infty} c_n\pair{x-x_0}^n\tag{$c_n\coloneqq \frac{a_{n-1}}{n}$}\\
            \limsup_{\ntoinf} \pair{\abs{c_n}}^{\frac{1}{n}} &= \limsup_{\ntoinf} \abs{\frac{a_{n-1}}{n}}^{\frac{1}{n}}\\
            \intertext{Es gilt}
            \pair{\frac{\abs{a_{n-1}}}{n}}^{\frac{1}{n}} &= \frac{1}{n^{\frac{1}{n}}}\pair{\abs{a_{n-1}}^{\frac{1}{n-1}}}^{\frac{n-1}{n}}\\
            \impl \limsup_{\ntoinf} \abs{c_n}^{\frac{1}{n}} &= \limsup_{\ntoinf} \abs{a_n}^{\frac{1}{n}}
            \intertext{Das heißt $F$ hat denselben Konvergenzradius wie $f$. Unsere Hoffnung ist also, dass $F$ eine Stammfunktion von $f$ ist oder}
            \int_{x_0}^{x} f\of{t} \dif t &= F\of{x}
            \intertext{Das gilt tatsächlich und lässt sich folgendermaßen zeigen. Wir definieren eine Funktionenfolge}
            f_n\of{x} &= \sum_{k=0}^{n} a_k\pair{x-x_0}^k
            \intertext{Wir wissen $\forall\delta > 0$ klein genug (konkret heißt das $\delta < R$) konvergiert $f_n$ gleichmäßig gegen $f$ auf dem Intervall $\interv{x_0-R+\delta, x_0+R-\delta}$. Dann gilt nach Satz~\ref{satz:gleichm-int} für $x\in\interv{x_0-R+\delta, x_0+R-\delta}$ fest}
            \int_{x_0}^{x} f\of{t} \dif t &= \lim_{\ntoinf} \int_{x_0}^{x} f_n\of{t} \dif t\\
            &= \lim_{\ntoinf} \int_{x_0}^{x} \sum_{k=0}^{n} \frac{a_k}{k+1}\pair{x-x_0}^{k+1} \dif x = F(x)\\
            \int_{x_0}^{x} f_n\of{t} \dif t &= \int_{x_0}^{x} \sum_{k=0}^{n} a_k\pair{x-x_0}^k \dif t = \sum_{k=0}^{n} a_k \int_{x_0}^{x} \pair{t-x_0}^k \dif t\\
            &= \interv{\frac{1}{k+1}\pair{t-x_0}^{k-1}}_{x_0}^{x} = \frac{1}{k+1}x{k+1}
        \end{align*}
    \end{beispiel}

    \begin{satz}
        Sei $I=\interv{a,b}$ sowie $f_n: I\fromto\R$ (oder $\C$) und die folgenden Voraussetzungen gelten
        \begin{enumerate}[label=(\roman*)]
            \item $\exists x_0\in I\colon f_n\of{x_0}$ konvergiert gegen $f\of{x_0}$
            \item $\pair{f_n'}_n$ konvergiert gleichmäßig gegen eine Funktion $g$
            \item $f_n'$ ist stetig für alle $n\in\N$
        \end{enumerate}
        Dann gilt $f(x) \coloneqq \displaystyle\lim_{\ntoinf} f_n\of{x}~\forall x\in I$ und $f$ ist stetig differenzierbar mit Ableitung $f' = g$.
        \begin{proof}
            Sei $x\in I$. Da alle Ableitungen von $f_n$ stetig sind, können wir den Hauptsatz verwenden und es gilt
            \begin{align*}
                f_n\of{x} - f_n\of{x_0} &= \int_{x_0}^{x} f_n'\of{t} \dif t\\
                \impl f_n\of{x} &= \underbrace{f_n\of{x_0}}_{\fromto f\of{x_0}} + \underbrace{\int_{x_0}^{x} f_n'\of{t} \dif t}_{\fromto \int_{x_0}^{x} g\of{t} \dif t}\\
                \impl f\of{x} &\coloneqq \lim_{\ntoinf} f_n\of{x} \text{ existiert } \forall x\in I \text{ und}\\
                f\of{x} &= f\of{x_0} + \int_{x_0}^{x} g\of{t} \dif t
            \end{align*}
            Nach dem Hauptsatz gilt, dass $f$ stetig differenzierbar ist mit $f' = g$.
        \end{proof}
    \end{satz}

    \begin{anwendung}
        \label{anwendung:potenzreihe-diff}
        \begin{align*}
            f\of{x} &= \sum_{n=0}^{\infty} a_N\pair{x-x_0}^{n}\\
            R &= \frac{1}{\displaystyle \limsup_{\ntoinf} \abs{a_n}^{\frac{1}{n}}} > 0\\
            f_n\of{x} &= \sum_{k=0}^{n} a_k\pair{x-x_0}^k\\
            \impl f\of{x} &= \lim_{\ntoinf} f_n\of{x}\\
            f_n'\of{x} &= \sum_{k=1}^{\infty} k\cdot a_k\pair{x-x_0}^{k-1}
            \intertext{Es gilt}
            \limsup_{\ntoinf} \abs{\pair{n+1} a_{n+1}}^{\frac{1}{n}} &= \limsup_{\ntoinf} \abs{a_{n+1}}^{\frac{1}{n+1}}
            \intertext{Nach dem vorherigen Satz gilt damit}
            f_n'\of{x} &= \sum_{k=1}^{n} k\cdot a_k\pair{x-x_0}^{k-1}
            \intertext{konvergiert auch auf $\pair{x_0-R, x_0+R}$ und gleichmäßig auf $\interv{x_0-R+\delta, x_0+R-\delta}$. Also konvergiert}
            \sum_{n=0}^{\infty} a_n\pair{x-x_0}^n &= \of{x}
            \intertext{und ihre Ableitung ist gegeben durch}
            \sum_{n=1}^{\infty} n\cdot a_n\pair{x-x_0}^{n-1}
        \end{align*}
        Also ist jede Potenzreihe differenzierbar auf ihrem Konvergenzintervall.
    \end{anwendung}

    \begin{korollar}
        \label{korollar:potenzreihe-diffb}
        Jede Potenzreihe $f\of{x} = \sum_{n=0}^{\infty} a_n\pair{x-x_0}^{n}$ ist unendlich oft differenzierbar auf ihrem Konvergenzintervall.
        \begin{proof}
            Nach Anwendung~\ref{anwendung:potenzreihe-diff} ist eine Potenzreihe einmal differenzierbar mit einer Potenzreihe als Ableitung. Damit folgt induktiv die Behauptung. Insbesondere gilt
            \begin{align*}
                f'\of{x} &= \sum_{n=1}^{\infty} n a_n\cdot\pair{x-x_0}^{n-1}\\
                f''\of{x} &= \sum_{n=2}^{\infty} n\cdot\pair{n-1}\cdot a_n\cdot\pair{x-x_0}^{n-2}\\
                f^{(k)}\of{x} &= \sum_{n=k}^{\infty} n\cdot\pair{n-1}\cdot\ldots\cdot\pair{n-k+1}\cdot a_n\cdot \pair{x-x_0}^{n-k}\\
                \impl f^{(k)}\of{x_0} &= k!\cdot a_k\\
                \equivalent a_k &= \frac{f^{(k)}\of{x_0}}{k!}
            \end{align*}
        \end{proof}
    \end{korollar}

    \begin{beispiel}
        Wir wissen
        \begin{align*}
            \sum_{n=0}^{\infty} x^{n} &= \frac{1}{1-x}\tag{$\abs{x} < 1$}\\
            \impl \sum_{n=1}^{\infty} n\cdot x^{n} &= x\cdot \sum_{n=1}^{\infty} n\cdot x^{n-1}\\
            &= x\cdot \frac{\dif}{\dif x}\cdot \frac{1}{1-x}\\
            &= x\cdot \frac{\dif}{\dif x} \sum_{n=0}^{\infty} x^{n} = x\cdot \frac{-1}{\pair{1-x}^2}\pair{-1} = \frac{x}{\pair{1-x}^2}
        \end{align*}
    \end{beispiel}

    \begin{bemerkung}[Taylorrreihe]
        \begin{align*}
            f\of{x} - f\of{x_0} &= \int_{x_0}^{x} f'\of{t} \dif t\\
            \impl f\of{x} &= f\of{x_0} + \int_{x_0}^{x} f'\of{t} \dif t\\
            &= f\of{x_0} + \int_{x_0}^{x} \pair{f'\of{t} - f'\of{x_0} + f'\of{x_0}} \dif x\\
            &= f\of{x_0} + \int_{x_0}^{x} \pair{f'\of{t} - f'\of{x_0}} \dif t + f'\of{x_0} \cdot \int_{x_0}^{x} 1 \dif t\\
            &= f\of{x_0} + f'\of{x_0}\cdot \pair{x-x_0} + \underbrace{\int_{x_0}^{x} \pair{f'\of{t} - f'\of{x_0}} \dif t}_{\eqqcolon R_{x_0}\of{x}}
            \intertext{Wir können den Fehler abschätzen und erhalten für ein $\varepsilon\of{x} \coloneqq \sup_{t\in\pair{x_0, x}} \abs{f'\of{t} - f'\of{x_0}}$}
            \abs{R_{x_0}\of{x}} &\leq \int_{x_0}^{x} \abs{f'\of{t} - f'\of{x_0}} \dif t \leq \varepsilon\of{x}\cdot\abs{x-x_0}\\
            \frac{\abs{R_{x_0}\of{x}}}{\abs{x-x_0}} &= \varepsilon\of{x} \fromto 0 \text{ für } x\fromto x_0
        \end{align*}
    \end{bemerkung}

    \newpage


    \section{[*] Taylors Theorem}
    \imaginarysubsection{Taylors Theorem}
    \thispagestyle{pagenumberonly}

    \begin{align*}
        f\of{x} &= f\of{x_0} + \int_{x_0}^{x} f'\of{t} \dif t\numberthis\label{eq:taylor}
    \end{align*}

    \begin{satz} % Satz 1
        \marginnote{[28. Mai]}
        \label{satz:taylor}
        Sei $f\in\mC^{(n+1)}\of{\pair{a,b}}$ ($n+1$ mal stetig differenzierbar auf $\pair{a,b}$). Dann gilt für alle $x, x_0\in\pair{a,b}$
        \begin{align*}
            f\of{x} &= f\of{x_0} + f'\of{x_0}\pair{x-x_0} + \frac{f''\of{x_0}}{2}\pair{x-x_0}^2\\
            &\quad+ \dots + \frac{f^{(n)}\of{x_0}}{n!}\pair{x-x_0}^n + R_n\of{f, x_0, x}
            \intertext{mit}
            R_n\of{f, x_0, x} &= \frac{1}{n!} \int_{x_0}^{x} \pair{x-t}^{n}f^{(n+1)}\of{t} \dif t
        \end{align*}
        \begin{proof}
            Wir verwenden Induktion. Der Induktionsanfang für $n=1$ ist gerade der Hauptsatz.\\[.5\baselineskip]
            Induktionsschritt: Angenommen $f\in\mC^{(n+2)}$. Dann gilt nach Induktionsannahme
            \begin{align*}
                f\of{x} &= \sum_{k=0}^{n} \frac{f^{(k)}\of{x_0}}{k!}\pair{x-x_0}^{k} + R_n\of{f, x_0, x}\tag{1}\\
                R_n\of{f, x_0, x} &= \frac{1}{n!} \int_{x_0}^{x} \pair{x-t}^{n}f^{(n+1)}\of{t} \dif t
                \intertext{Wir integrieren partiell}
                &= \frac{1}{n!}\pair{\interv{-\frac{1}{n+1}\pair{x-t}^{n+1}f^{(n+1)}\of{t}}_{x_0}^x - \int_{x_0}^{x} \frac{1}{n+1}\pair{x-t}^{n+1}f^{(n+2)}\of{t} \dif t}\\
                &= -\frac{1}{n+1}\cdot\frac{\dif}{\dif t}\pair{x-t}^{n+1}
                \intertext{Nach (1) folgt}
                f\of{x} &= \sum_{k=0}^{n+1} \frac{f^{(k)}\of{x_0}}{k!}\pair{x-x_0}^k + \underbrace{\frac{1}{\pair{n+1}!} \int_{x_0}^{x} \pair{x-t}^{n+1}\cdot f^{(n+2)}\of{t} \dif t}_{=R_{n+1}\of{f, x_0, x}}\qedhere
            \end{align*}
        \end{proof}
    \end{satz}

    \begin{korollar} % Korollar 1
        Sei $f\in\mC^n\of{\pair{a,b}}$. Dann gilt $\forall x, x_0\in\pair{a,b}$:
        \begin{align*}
            f\of{x} &= \sum_{k=0}^{n} \frac{f^{(k)}\of{x_0}}{k!}\pair{x-x_0}^k + \overline{R}_n\of{f, x_0, x}
            \intertext{mit}
            \ov{R}_n\of{f, x_0, x} &= \frac{1}{\pair{1-n}!} \int_{x_0}^{x} \pair{x-t}^{n-1}\cdot\interv{f^{(n)}\of{t} - f^{(n)}\of{x}} \dif t
        \end{align*}
        \begin{proof}
            Nach Satz~\ref{satz:taylor} gilt
            \begin{align*}
                f(x) &= \sum_{k=0}^{n-1} \frac{f^{(n)}\of{x_0}}{k!}\cdot\pair{x-x_0}^k + \ov{R}_{n-1}\of{f, x_0, x}\\
                &= \sum_{k=0}^{n} \frac{f^{(k)}\of{x_0}}{k!}\pair{x-x_0}^k + R_{n-1}\of{f, x_0, x} - \frac{f^{(n)}\of{x_0}}{n!}\pair{x-x_0}^{n}\\
                \pair{n-1}!\cdot R_{n-1}\of{f, x_0, x} &= \int_{x_0}^{x} \pair{x-t}^{n-1}f{(n)}\of{t} \dif t - \frac{1}{n}f^{(n)}\of{x_0}\pair{x-x_0}^n\\
                &= \int_{x_0}^{x} \pair{x-t}^{n-1}\cdot\interv{f^{(n)}\of{t} - f^{(n)}\of{x_0}} \dif t\qedhere
            \end{align*}
        \end{proof}
    \end{korollar}

    \begin{bemerkung}
        \begin{align*}
            n!\cdot\abs{\frac{R_n\of{f, x_0, x}}{\pair{x-x_0}^n}} &= \abs{\int_{x_0}^{x} \frac{\pair{x-t}^n}{\pair{x-x_0}^n}f^{(n+1)}\of{t} \dif t}\\
            &\leq \int_{x_0}^{x} \abs{\frac{x-t}{x-x_0}}^n\cdot\abs{f^{(n+1)}\of{t}} \dif t\\
            &\leq \int_{x_0}^{x} \abs{f^{(n+1)}\of{t}} \dif t\fromto 0
        \end{align*}
    \end{bemerkung}

    \begin{definition}
        Sei $f\in\mC\of{\pair{a,b}}$ und $x_0\in\pair{a,b}$. Wir definieren
        \begin{align*}
            T_n\of{f, x_0}\of{x} &\coloneqq \sum_{k=0}^{n} \frac{f^{(k)}\of{x_0}}{k!}\pair{x-x_0}^k \tag{$n$-tes Taylorpolynom}
            \intertext{Ist $f$ unendlich oft differenzierbar, so nennen wir}
            T\of{f, x_0, x} &= \sum_{n=0}^{\infty} \frac{f^{(n)}\of{x_0}}{n!}\pair{x-x_0}^n
        \end{align*}
        Taylorreihe (von $f$ im Entwicklungspunkt $x_0$).
    \end{definition}

    \begin{bemerkung}
        \theoremescape
        \begin{enumerate}[label=(\roman*)]
            \item Die Taylorreihe kann Konvergenzradius $R > 0$ haben
            \item Ist eine Taylorreihe konvergent, so muss sie nicht unbedingt gegen $f$ konvergieren
        \end{enumerate}
    \end{bemerkung}

    \begin{beispiel}
        Wir betrachten $f: \R\fromto\R$
        \begin{align*}
            x\mapsto f\of{x}&\coloneqq \begin{cases}
                                           e^{-\frac{1}{x^2}} & x\neq 0\\
                                           0 &x=0
            \end{cases}
        \end{align*}
        Dann ist $f$ unendlich oft differenzierbar und es gilt $f^{(n)}\of{0} = 0~\forall n\in\N_0$.
        \begin{proof}
            \textsc{Schritt 1}: Sei $x\neq 0$. Dann existiert $\forall n\in\N_0$ ein Polynom $p_n$, sodass
            \begin{align*}
                f^{(n)}\of{x} &= p_n\of{\frac{1}{n}}\cdot e^{-\frac{1}{x^2}}
            \end{align*}
            Wir beweisen diese Behauptung mittels Induktion.~\\
            \begin{induktionsanfang}
                Es ist $n=0$. Wir wählen $p_0\of{x} = 1$.
            \end{induktionsanfang}
            \begin{induktionsschritt}
                \begin{align*}
                    f^{(n+1)}\of{x} &= \frac{\dif}{\dif x}\pair{f^{(n)}\of{x}}\\
                    &= \frac{\dif}{\dif x}\pair{p_n\of{\frac{1}{x}}\cdot e^{-\frac{1}{x^2}}}\\
                    &= p_n'\of{\frac{1}{x}}\cdot\pair{-\frac{1}{x^2}}\cdot e^{-\frac{1}{x^2}} + p_n\of{\frac{1}{x}}\cdot e^{-\frac{1}{x^2}}\cdot\frac{2}{x^3}\\
                    &= \underbrace{\pair{-p_n'\of{\frac{1}{x}}\cdot\frac{1}{x^2} + 2p_n\of{\frac{1}{x}}\cdot\frac{1}{x^3}}}_{\definedasbackwards p_{n+1}\of{\frac{1}{x}}}\cdot e^{-\frac{1}{x^2}}\\
                    p_{n+1}\of{t} &\coloneqq -p'_n\of{t}\cdot t^2 + 2t^3\cdot p_n\of{t}
                \end{align*}
            \end{induktionsschritt}~\\
            \textsc{Schritt 2}: $f^{(n)}\of{0} = 0~\forall n\in\N_0$. Wir nutzen wieder Induktion. Der Induktionsanfang ist klar.
            \begin{induktionsschritt}
                Angenommen $f^{(n)}\of{0} = 0$. Dann gilt
                \begin{align*}
                    f^{(n+1)}\of{0} &= \lim_{x\fromto 0} \frac{f^{(n)}\of{x} - f^{(n)}\of{0}}{x}\\
                    &= \lim_{x\fromto 0} \frac{f^{(n)}\of{x}}{x}\\
                    &= \lim_{x\fromto 0}\pair{\frac{1}{x}\cdot p_n\of{\frac{1}{x}}\cdot e^{-\frac{1}{x^2}}}\\
                    &= \lim_{\abs{R}\fromto \infty} \pair{R\cdot p_n\of{R}\cdot e^{-R^2}} = 0\qedhere
                \end{align*}
            \end{induktionsschritt}
        \end{proof}
    \end{beispiel}

    \begin{satz} % Satz 4
        Ist $f\of{x} = \sum_{n=0}^{\infty} a_n\pair{x-x_0}^{n}$ eine Potenzreihe mit Konvergenzradius $r>0$, so ist die Taylorreihe von $f$ gleich dieser Potenzreihe.
        \begin{proof}
            Folgt aus Korollar~\ref{korollar:potenzreihe-diffb} und Gleichung ??.
        \end{proof}
    \end{satz}

    \begin{beispiel}
        \begin{align*}
            \sum_{n==}^{\infty} \frac{\pair{cx}^n}{n!} &= \sum_{n=0}^{\infty} \frac{c^n}{n!}\cdot x^n\\
            \exp\of{cx} &= \exp\of{cx_0 + c\pair{x-x_0}}\\
            &= \exp\of{cx_0}\cdot\exp\of{c\pair{x-x_0}}\\
            &= \exp\of{cx_0}\cdot \sum_{n=0}^{\infty} \frac{c^n}{n!}\pair{x-x_0}^n\\
            &= \sum_{n=0}^{\infty} ??
        \end{align*}
    \end{beispiel}

    \begin{satz}[Restglieddarstellung von Schlömilch] % Satz 5
        \label{satz:restglied-schloemilch}
        Sei $f\in\mC^{n+1}\of{\pair{a,b}}$ und $x_0\in\pair{a,b}$. Dann gilt
        \begin{align*}
            f\of{x} &= T_n\of{f, x_0, x} + R_n\of{f, x_0, x}
            \intertext{mit}
            ????
        \end{align*}
    \end{satz}

    \begin{bemerkung}
        Ist $p=n+1$, dann haben wir die Lagrangsche Darstellung
        \begin{align*}
            R_n\of{f, x_0, x} &= \frac{1}{\pair{n+1}!}\cdot f^{(n+1)}\of{\xi}\cdot\pair{x-x_0}^{n+1}\of{\xi}
            \intertext{und wenn $p=1$, dann haben wir die Cauchysche Darstellung}
            R_n\of{f, x_0, x} &= \frac{1}{n!}\cdot f^{(n+1)}\of{\xi}\cdot\pair{x-\xi}^n\cdot\pair{x-x_0}
        \end{align*}
        für das Restglied.
    \end{bemerkung}

    \begin{satz}[Logarithmus] % Satz 6
        Für die Logarithmusreihe $f_n: -1\leq x \leq 1$ gilt
        \begin{align*}
            \log\of{1+x} &= x-\frac{x^2}{2} + \frac{x^2}{3} \pm \dots = \sum_{n=1}^{\infty} \pair{-1}^{n+1}\cdot \frac{x^n}{n}
        \end{align*}

        \begin{proof}
            \begin{align*}
                f\of{x} &= \log\of{1+x}\\
                f'\of{x} &= \pair{1+x}^{-1}\\
                f''\of{x} &= -1\cdot\pair{1+x}^{-2}\\
                \vdots\\
                f^{(n)}\of{x} &= \pair{-1}^{n+1}\cdot\pair{n-1}!\cdot \pair{1+x}^{-n}\\
                T_n\of{f, 0}\of{x} &= \sum_{k=0}^{n} \frac{f^{(k)}\of{0}}{k} \cdot x^k = \sum_{k=0}^{n} \pair{-1}^{n+1}\cdot \frac{\pair{k-1}}{k!}x^k\\
                &= \sum_{k=0}^{n} \pair{-1}^{k+1}\cdot \frac{x^k}{k}
                \intertext{\textsc{Schritt 1}: Aus Satz~\ref{satz:restglied-schloemilch} folgt}
                f\of{x} &= \sum_{k=1}^{n} \pair{-1}^{k+1}\cdot \frac{x^k}{k} + R_n\of{f, 0, x}\\
                R_n\of{f, 0, x} &= \frac{1}{pn!}\cdot f^{(n+1)}\of{x}\cdot\pair{x-\xi}^{n+1-p}\cdot\pair{x-x_0}^p\\
                &= n!\cdot\pair{-1}^{n+1}\cdot\pair{1+\xi}^{-\pair{n+1}}\\
                \impl \abs{R_n\of{f, 0, x}} &= \frac{1}{pn!}\cdot n! \cdot\pair{1+\xi}^{-n-1}\cdot\abs{x-\xi}^{n+1-p}\cdot\abs{x}^p
                \intertext{Angenommen $0\leq x \leq 1$. $0 < \xi < x$, wir wählen $p=n+1$}
                \impl \abs{R_n\of{f, 0, x}}&\leq \frac{1}{p} = \frac{1}{n+1}\fromto 0\marginnote{[31. Mai]}
                \intertext{Angenommen $-1\leq x \leq 0$. Dann gibt es ein $\xi$ zwischen 0 und $x$, das heißt $\xi = \Theta x$ mit $0 < \Theta < 1$. Dann gilt}
                R_n\of{f, 0, x} &= \frac{1}{p}\cdot\pair{-1}^n\cdot\pair{1+\Theta x}^{-(n+1)}\cdot\pair{x-\Theta x}^{n+1-p}\cdot x^p\\
                \impl \abs{R_n\of{f, 0, x}} &= \frac{1}{p} \cdot \pair{1+\Theta x}^{-(n+1)}\cdot\abs{x}^{n+1-p}\cdot\pair{1-\Theta}^{n+1-p}\cdot\abs{x}^p\\
                &= \frac{1}{p}\cdot \pair{1+\Theta x}^{-(n+1)}\cdot \pair{1-\Theta}^{n+1-p}\cdot \abs{x}^{n+1}
                \intertext{Da $-1\leq x \leq 0$}
                \impl 1 + \Theta x &= 1 - \Theta\cdot\abs{x} \geq 1 - \Theta > 0\\
                \impl \pair{1+\Theta x}^{-n} &\leq \pair{1-\Theta}^{-n}\\
                \impl \abs{R_n\of{f, 0, x}} &\leq \frac{1}{p}\cdot\pair{1-\Theta}^{-n}\cdot\pair{1-\abs{x}}^{-1}\cdot\pair{1-\Theta}^{n+1-p}\cdot\abs{x}^{n+1}
                \intertext{Wähle $p=1$}
                \impl \abs{R_n\of{f, 0, x}} &\leq \pair{1-\Theta}^{-n}\cdot\pair{1-\Theta}^{n} \cdot \frac{\abs{x}^{n+1}}{1-\abs{x}}= \frac{\abs{x}^{n+1}}{1-\abs{x}}\fromto 0
                \intertext{\textsc{Schritt 2}: Wir wollen zeigen, dass die Taylorreihe $ \sum_{n=1}^{\infty} \frac{(-1)^{n+1}}{n}\cdot x^n$ für alle $-1\leq x \leq 1$ konvergiert. Für $-1\leq x \leq 0$ gilt}
                \abs{\frac{(-1)^{n+1}}{n}\cdot x^n} &\leq \frac{1}{n}\cdot\abs{x}^n\leq \abs{x}^n
            \end{align*}
            Damit folgt die Konvergenz aus dem Vergleich mit der geometrischen Reihe. Das gleiche Prinzip lässt sich für $0\leq x< 1$ anwenden. Für $x=1$ ist $ \sum_{n=1}^{\infty} \frac{(-1)^{n+1}}{n}$ eine alternierende monotone Reihe, die damit nach Leibniz konvergiert.\\[.2\baselineskip]
            Aus \textsc{Schritt 1} und \textsc{Schritt 2} folgt damit die Behauptung.
        \end{proof}
    \end{satz}

    \begin{korollar}
        Für $a > 0$ und $0< x \leq 2a$ folgt
        \begin{align*}
            \log x &= \log a + \sum_{n=1}^{\infty} \frac{\pair{-1}^{n+1}}{n\cdot a^n}\pair{x-a}^n
        \end{align*}
        \begin{proof}
            \begin{align*}
                \log x &= \log\of{a+\pair{x-a}} = \log\of{a\cdot\pair{1+\frac{x}{a}}} \\
                &= \log a + \log\of{1+\frac{x}{a}}\qedhere
            \end{align*}
        \end{proof}
    \end{korollar}

    \begin{bemerkung}
        Es gilt
        \begin{align*}
            \log 2 &= \log\of{1+1} = \sum_{n=1}^{\infty} \frac{(-1)^{n+1}}{n}\tag{konvergiert langsam}\\
            \log\of{1+x} &= \sum_{n=1}^{\infty} \frac{(-1)^{n+1}}{n}\cdot x^n\\
            \log\of{1-x} &= \sum_{n=1}^{\infty} \frac{(-1)^{n+1}}{n}\cdot (-1)^n\cdot x^n = - \sum_{n=1}^{\infty} \frac{x^n}{n}\\
            \impl \log\of{1+x} - \log\of{1-x} &= \sum_{n \text{ ungerade}}^{} \pair{\frac{x^n}{n}+\frac{x^n}{n}} = 2\cdot \sum_{k=0}^{\infty} \frac{x^{2k+1}}{2k+1}= \log\of{\frac{1+x}{1-x}}
            \intertext{Für ein $y>1$ mit $y=\frac{1+x}{1-x}$ gilt}
            \pair{1-x}\cdot y &= 1+x\\
            \equivalent y-1 &= x\cdot\pair{y+1}\\
            \equivalent x &= \frac{y-1}{y+1}
            \intertext{Für $y=2$ gilt also $x=\frac{1}{3}$. Das heißt}
            \log y &= 2\cdot \sum_{k=0}^{\infty} \frac{1}{2k+1}\cdot\pair{\frac{y-1}{y+1}}^{2k+1}\\
            \impl \log 2 &= 2\cdot \sum_{k=0}^{\infty} \frac{1}{2k+1}\cdot\pair{\frac{1}{3}}^{2k+1}\tag{konvergiert schneller}
        \end{align*}
    \end{bemerkung}

    \begin{satz}[Abelscher Grenzwertsatz] % Satz 8
        \label{satz:abel-grenzwert}
        Angenommen $\displaystyle \sum_{n=0}^{\infty} a_n$ konvergiert. Dann ist die Potenzreihe $\displaystyle f(x) \coloneqq\sum_{n=0}^{\infty} a_n\cdot x^n$
        \begin{enumerate}[label=(\roman*)]
            \item konvergent für alle $-1 < x \leq 1$
            \item stetig in $x=1$ und
            \item Die Potenzreihe konvergiert gleichmäßig auf allen Intervallen $\interv{a, 1}$ mit $-1 < a < 1$. (Das heißt sie konvergiert lokal gleichmäßig auf $\interv{-1, 1}$). Insbesondere in jeder $\varepsilon$-Umgebung um $x=1$.
        \end{enumerate}

        \begin{proof}
            \textsc{Schritt 1}: Wir zeigen zunächst (ii) und setzen dafür
            \begin{align*}
                A_n &\coloneqq \sum_{k=n+1}^{\infty} a_{k}\fromto 0 \text{ für } \ntoinf
                \intertext{Insbesondere ist}
                \sup_{n\geq 0} \abs{A_n} &< \infty\\
                \impl \sup_{n\geq k+1} \abs{A_n} &\fromto 0 \text{ für } k\toinf\\
                a_n &= A_{n-1} - A_n\tag{Wir setzen $A_{-1} = \sum_{n=0}^{\infty} a_n$}
                \intertext{Für ein $L\in\N$ gilt}
                \sum_{n=0}^{L} a_n \cdot x^n &= \sum_{n=0}^{L} \pair{A_{n-1}-A_n}\cdot x^n\\
                &= \sum_{n=0}^{L} A_{n-1}\cdot x^n - \sum_{n=0}^{L} A_n\cdot x^n\\
                &= \sum_{j=-1}^{L-1} A_j\cdot x^{j+1} - \sum_{j=0}^{L} A_j \cdot x^j\\
                &= A_{-1}\cdot x^0 - A_{L}\cdot x^L + \sum_{n=0}^{L} A_n\cdot\pair{x^{n+1}-x^n}\\
                &= f\of{1} - A_{L}\cdot x^L + \pair{x-1}\cdot \sum_{n=0}^{L-1} A_n \cdot x^n
                \intertext{Es gilt $\abs{A_L\cdot x^L} \leq \abs{A_L}$ und $\abs{A_n}\leq C$ für eine Konstante $C$. Das heißt für $\abs{x} < 1$}
                \impl \sum_{n=0}^{\infty} A_n\cdot x^n &\text{ hat Limes für }L\toinf\\
                \impl f\of{x} &= \lim_{L\toinf} \sum_{n=0}^{L} a_n\cdot x^n = f\of{1}+ \pair{x-1}\cdot \sum_{n=0}^{\infty} A_n\cdot x^n\\
                \impl \abs{f\of{1} - f\of{x}} &= \pair{1-x} \cdot \abs{\sum_{n=0}^{\infty} A_n \cdot x^n} \leq \pair{1-x} \cdot \sum_{n=0}^{\infty} \abs{A_n}\cdot x^n
                \intertext{Sei $K\in\N$. Dann gilt}
                \impl \abs{f\of{1} - f\of{x}} &\leq \pair{1-x}\cdot \sum_{n=0}^{K} \abs{A_n}\cdot x^n + \pair{1-x} \cdot \sum_{n=K+1}^{\infty} \abs{A_n}\cdot x^n\\
                &\leq \underbrace{\pair{1-x} \cdot \sup_{n\geq 0}\of{\abs{A_n} }\cdot \sum_{n=0}^{K} x^n}_{\definedasbackwards I_{K}\of{x}} + \underbrace{\pair{1-x}\cdot \sup_{n\geq K+1}\of{\abs{A_n}}\cdot \sum_{n=K+1}^{\infty} x^n}_{\definedasbackwards J_{K}\of{x}}
                \intertext{Für ein festes $K\in\N$ geht $I_{K}\fromto 0$ für $x\fromto 1-$ und es gilt}
                J_{K}\of{x} &= \sup_{n\geq K + 1}\of{\abs{A_n}}\cdot\pair{1-x}\cdot \sum_{n=K+1}^{\infty} x^n
                \intertext{Nach der geometrischen Summenformel gilt}
                &= \sup_{n\geq K + 1}\of{\abs{A_n}}\cdot \pair{1-x}\cdot\frac{x^{K+1}}{1-x}\\
                &\leq \sup_{n\geq K + 1}\of{\abs{A_n}}\fromto 0 \text{ für } L\toinf \tag{gleichmäßig in $0\leq x <  1$}\\
                \impl \limsup_{x\fromto 1-} \abs{f\of{1} - f\of{x}} &\leq 0 + \limsup_{x\fromto 1-} J_K\of{x}\\
                &\leq \sup_{n\geq K+1}\of{\abs{A_n}}\fromto 0 \text{ für } K\toinf\quad\forall K\in\N\\
                \impl \limsup_{x\fromto 1-} \abs{f\of{1}-f\of{x}} &= 0\\
                \impl \lim_{x\fromto 1-} f\of{x} &= f\of{1}
                \intertext{\textsc{Schritt 2}: $f_n\of{x} = \sum_{k=0}^{n} a_k\cdot x^k$}
                \impl f\of{x} - f_n\of{x} &= \pair{x-1}\cdot \sum_{k=n+1}^{\infty} A_k \cdot x^k - A_n\cdot x^n\\
                \impl \abs{f\of{x} - f_n\of{x}} &\leq \pair{1-x} \cdot \sum_{k=n+1}^{\infty} \abs{A_k}\cdot x^k + \abs{A_n}\cdot x^n\tag{$0\leq x < 1$}\\
                &\leq \pair{1-x} \cdot \sup_{k\geq n+1}\of{\abs{A_k}}\cdot \sum_{k=n+1}^{\infty} x^k + \abs{A_n}\\
                &\leq \sup_{k\geq n+1}\of{\abs{A_k}} \cdot\pair{1-x}\cdot x^{n+1}\cdot \sum_{k=0}^{\infty} x^k + \abs{A_n}\\
                &\leq 2\cdot \sup_{k\geq n}\of{\abs{A_k}}
                \intertext{Mit (ii) folgt $\fa 0\leq x \leq 1$}
                \abs{f\of{x} - f_n\of{x}} &\leq 2\cdot \sup_{k\geq n}\of{\abs{A_k}}\\
                \impl \sup_{0 \leq x \leq 1}\of{\abs{f\of{x}-f_n\of{x}}} &\leq 2\cdot \sup_{k\geq n}\of{\abs{A_k}}
            \end{align*}
            Das heißt $(A_n)_n$ ist eine Nullfolge. Damit gilt gleichmäßige Konvergenz auf $\interv{0,1}$.\\
            $f\of{x}$ konvergiert gleichmäßig auf kompakten Teilintervallen innerhalb des Konvergenzradius und $ \sum_{}^{} a_n$ konvergiert mit Konvergenzradius $R\geq 1$. Das heißt $f\of{x}$ konvergiert gleichmäßig auf allen $\interv{-\delta, \delta}$ für $0<\delta < 1$.
        \end{proof}
    \end{satz}

    \begin{satz}[Arctan Reihe] % Satz 8?
        Für $\abs{x} \leq 1$ gilt
        \begin{align*}
            \arctan x &= x- \frac{x^3}{3} + \frac{x^5}{5} \pm \dots\\
            &= \sum_{n=0}^{\infty} \pair{-1}^n \cdot \frac{x^{2n+1}}{2n+1}
        \end{align*}
        \begin{proof}
            Es sei $f\of{x} = \arctan x$. Dann gilt
            \begin{align*}
                f'\of{x} &= \frac{1}{1+x^2} = \frac{1}{1-\pair{-x}^2}\\
                &= \sum_{n=0}^{\infty} \pair{-x^2}^n = \sum_{n=0}^{\infty} \pair{-1}^n \cdot x^{2n}
                \intertext{Nach dem Hauptsatz gilt}
                f\of{x} &= f\of{0} + \int_{0}^{x} f'\of{t} \dif t\\
                &= 0 + \int_{0}^{x} \frac{1}{1+t^2} \dif t\\
                &= \int_{0}^{x} \sum_{n=0}^{\infty} \pair{-1}^n\cdot t^{2n}  \dif t\\
                &= \sum_{n=0}^{\infty} \pair{-1}^{2n}\cdot \int_{0}^{x} t^{2n} \dif t\\
                &= \sum_{n=0}^{\infty} \pair{-1}^n\cdot \frac{x^{2n+1}}{2n+1} \text{ falls } \abs{x} < 1
            \end{align*}
            Für $x=1$ gilt
            \begin{align*}
                f\of{x} &= \sum_{n=0}^{\infty} \pair{-1}^n \cdot \frac{x^{2n+1}}{2n+1}\\
                f\of{1} &= \sum_{n=0}^{\infty} \pair{-1}^n \cdot \frac{1^{2n+1}}{2n+1}\\
                &= \sum_{n=0}^{\infty} \frac{(-1)^n}{2n+1} \text{ konvergiert nach Leibniz }
            \end{align*}
            Das heißt aus Satz~\ref{satz:abel-grenzwert} folgt die gleichmäßige Konvergenz von dieser Reihe für alle $\abs{x} \leq 1$.\\
            Das heißt aus der Stetigkeit von $\arctan$ bei $\pm 1$ und dem Satz folgt
            \begin{align*}
                \arctan x &= \sum_{n=0}^{\infty} \pair{-1}^n\cdot \frac{x^{2n+1}}{2n+1}\quad\forall \abs{x}\leq 1\qedhere
            \end{align*}
        \end{proof}
    \end{satz}

    \begin{bemerkung}[Reihendarstellung von $\pi$]
        \marginnote{[04. Jun]}
        Es gilt $\tan x = \frac{\sin x}{\cos x}$ und damit $1=\tan \frac{\pi}{4}$, $\arctan 1 = \frac{\pi}{4}$. So ergibt sich mit dem Arctan eine Reihendarstellung von $\pi$
        \begin{align*}
            \frac{\pi}{4} &= \sum_{n=0}^{\infty} \frac{\pair{-1}^n}{2n+1} = 1 - \frac{1}{3} + \frac{1}{5} - \frac{1}{7} + \dots
            \intertext{Diese Reihe konvergiert für die tatsächliche Anwendung allerdings zu langsam. Viel schneller ist die Berechnung über die \emph{Machinsche Formel}}
            \frac{\pi}{4} &= 4\cdot\arctan \frac{1}{5} - \arctan \frac{1}{239}
        \end{align*}
    \end{bemerkung}

    \begin{satz}[Binomische Reihe]
        Sei $\alpha\in\R$. Dann gilt für $\abs{x} < 1$\footnote{Gefunden von Newton 1665}
        \begin{align*}
            \pair{1+x}^{\alpha} &= \sum_{n=0}^{\alpha} \binom{\alpha}{n} x^n
            \intertext{wobei wir den verallgemeinerten Binomialkoeffizient verwenden}
            \binom{\alpha}{n} &\coloneqq \prod_{k=1}^{n} \frac{\alpha - k +1}{k} = \frac{\alpha\cdot\pair{\alpha-1}\cdot\ldots\cdot\pair{\alpha-k+1}}{k!}\\
            \binom{\alpha}{n} &\coloneqq 0 \text{ für } n\geq \alpha + 1
            \intertext{Daraus folgt der speziellere Binomische Lehrsatz für $m\in\N$}
            \impl \pair{1+x}^m &= \sum_{n=0}^{m} \binom{m}{n}\cdot x^n
        \end{align*}
        \begin{proof}
            \textsc{Schritt 1}: Sei $f\of{x} = \pair{1+x}^{\alpha}$ für $x > -1$. Dann gilt
            \begin{align*}
                f'\of{x} &= \alpha\cdot\pair{1+x}^{\alpha-1}\\
                f''\of{x} &= \alpha\cdot\pair{\alpha-1}\cdot\pair{1+x}^{\alpha-2}\\
                \vdots\\
                f^{(n)}\of{x} &= \alpha\cdot\pair{\alpha-1}\cdot \ldots\cdot \pair{\alpha-n+1}\cdot\pair{1-x}^{\alpha-n}
                \intertext{Das heißt die Taylorreihe für $f$ in $0$ ist}
                T\of{f, 0}\of{x} &= \sum_{n=0}^{\infty} \frac{f^{(n)}\of{0}}{n!}\cdot x^n\\
                &= \sum_{n=0}^{\infty} \frac{\alpha\cdot\pair{\alpha-1}\cdot\ldots\cdot\pair{\alpha-n+1}}{n!}\cdot x^n = \sum_{n=0}^{\infty} \binom{\alpha}{n} \cdot x^n
                \intertext{\textsc{Schritt 2}: Wir wollen zeigen, dass die obige Taylorreihe konvergiert}
                a_n &\coloneqq \binom{a}{n}x^n\quad \abs{x} < 1\\
                \abs{\frac{a_{n+1}}{a_n}} &= \abs{\frac{\binom{\alpha}{n+1}\cdot x^{n+1}}{\binom{\alpha}{n}\cdot x^n}}\\
                &= \abs{x}\cdot \abs{\frac{\alpha-n}{n+1}} \underset{\ntoinf}{\fromto} \abs{x} < 1\\
                \impl \exists x < 1, N_0\in\N\colon \abs{\frac{a_{n+1}}{a_n}} &\leq x < 1 \quad\forall n\geq N_0\\
                \impl \sum_{n=0}^{\infty} a_n &= \sum_{n=0}^{\infty} \binom{\alpha}{n}x^n \text{ ist absolut konvergent }
                \intertext{\textsc{Schritt 3}: Der Restterm soll verschwinden. Sei $0 < \Theta < 1$ und $\xi=\Theta x$, sowie $1 \leq p \leq n+1$}
                R_n\of{f, 0, x} &= \frac{1}{p\cdot n!} \cdot f^{(n+1)}\cdot \pair{\Theta x}\cdot\pair{x-\Theta x}^{n+1-p}\cdot x^p\tag{Schlömilch}
                \intertext{Für $p=1$ ergibt sich die Restglieddarstellung von Cauchy}
                R_n\of{f, 0, x} &= \frac{1}{n!}\cdot \alpha\cdot\pair{\alpha-1}\cdot\ldots\cdot \pair{\alpha - n+1}\cdot\pair{\alpha-n}\cdot \pair{1+\Theta x}^{-n-1}\cdot\pair{x-\Theta x}^n\cdot x\\
                &= \binom{\alpha}{n+1}\cdot x^{n+1}\cdot\pair{1-\Theta}^n\cdot\pair{1+\Theta x}^{-(n+1)}\\
                \impl \abs{R_n\of{f, 0, x}} &= \underbrace{\abs{\binom{\alpha}{n+1}\cdot x^{n+1}}}_{\fromto 0\text{ nach \textsc{Schritt 2}}} \cdot\pair{1-\Theta}^{n}\cdot \pair{1+\Theta x}^{-n-1}\\
                \pair{1+\Theta x}^{-(n+1)} &= \frac{1}{\pair{1+\Theta x}^{-n+1}}\\
                &= \frac{1}{1+\Theta x}\cdot \frac{1}{\pair{1+\Theta x}^{n}}\\
                &\leq \frac{1}{1-\abs{x}}\cdot \frac{1}{\pair{1-\Theta}^n}\\[5pt]
                \impl \abs{R_n\of{f, 0, x}} &\leq \abs{\binom{\alpha}{n+1}\cdot x^{n+1}}\cdot \frac{1-\Theta^n}{1-\Theta}\cdot \frac{1}{1-\abs{x}}\\
                &= \abs{\binom{\alpha}{n+1}\cdot x^{n+1}} \cdot \frac{1}{1-\abs{x}} \underset{\ntoinf}{\fromto}
                \intertext{Das heißt nach dem Satz von Taylor gilt}
                f\of{x} &= \sum_{n=0}^{\infty} \binom{\alpha}{n}\cdot x^n\qedhere
            \end{align*}
        \end{proof}
    \end{satz}

    \newpage


    \section{[*] Die Gamma-Funktion}

    \imaginarysubsection{Die $\Gamma$-Funktion}
    \thispagestyle{pagenumberonly}

    Erinnerung: Die $\Gamma$-Funktion ist für $x>0$ definiert als
    \begin{align*}
        \Gamma\of{x} &\coloneqq \int_{0}^{\infty} t^{x-1}\cdot e^{-t} \dif t
        \intertext{Das funktioniert bei $0$; da $-x - 1 > -1$ und es funktioniert bei $\infty$, da}
        t^{x-1}\cdot e^{-t} &= t^{x-1}\cdot e^{-\frac{t}{2}}\cdot e^{-\frac{t}{2}}\\
        &= C_x\cdot e^{-\frac{t}{2}} \tag{$C_x \coloneqq \sup_{t \geq 1} t^{x-1}\cdot e^{-\frac{t}{2}} < \infty$}\\
        \impl \Gamma\of{x} &= \lim_{a\searrow 0}\lim_{b\toinf} \int_{a}^{b} t^{x-1}\cdot e^{-t} \dif t \text{ existiert } \forall x>0
        \intertext{Wir hatten außerdem}
        \Gamma\of{x+1} &= x\cdot\Gamma\of{x}\quad\forall x > 0
    \end{align*}

    \begin{definition}[Konvexität\footnote{Siehe auch Skript Ana I, Kapitel 19}]
        Eine Funktion $F: I\fromto \R$ -- wobei $I$ ein Intervall ist ($I=\linterv{0, \infty}$ ist dabei erlaubt) -- heißt \emph{konvex}, falls $\forall x,y\in I$ und für alle $0\leq \Theta \leq 1$ gilt
        \begin{align*}
            F\of{\Theta x + \pair{1-\Theta}y} &\leq \Theta F\of{x} + \pair{1-\Theta}F\of{y}
        \end{align*}
    \end{definition}

    \begin{skizze}[Konvexe Funktion]
        Wähle ein $\Theta\in\pair{0,1}$ und formuliere die Interpolation $\Theta x + \pair{1-\Theta}\cdot y$.
        \begin{figure}[H]
            \centering
            \begin{tikzpicture}
                \draw[->] (-1, 0) -- (3, 0);
                \draw[->] (0, -1) -- (0, 4);
                \draw (-0.95*0.5, 0.1) -- (-0.95*0.5, -0.1) node[below] {$x$};
                \draw (5.2*0.5, 0.1) -- (5.2*0.5, -0.1) node[below] {$y$};

                \fill (-0.95*.5,0.7875*.5) circle[radius=1.5pt] node[left] {$F(x)$};
                \fill (5.2*.5,5.4*.5) circle[radius=1.5pt] node[right] {$F(y)$};
                \draw[domain=-2:6, smooth, variable=\x] plot ({0.5*\x}, {(0.2*(\x-0.25)^2+0.5)*0.5}) node[anchor=east] {$F$};
                \draw[domain=-0.95:5.2, smooth, variable=\x, dashed] plot ({0.5*\x}, {(0.75*\x+1.5)*0.5});
            \end{tikzpicture}
            \caption{Konvexe Funktion mit eingezeichneter Sekante}
        \end{figure}
    \end{skizze}

    \begin{beispiel}
        Die Funktionen $F\of{t} = e^{t}$ und $F\of{t} = e^{-t}$ sind konvex auf $\R$. (Übung)
    \end{beispiel}

    \begin{definition}
        Eine Funktion $F$ heißt \emph{konkav}, falls $-F$ konvex ist. Das heißt
        \begin{align*}
            F\of{\Theta x + \pair{1-\Theta}y} &\geq \Theta F\of{x} + \pair{1-\Theta}F\of{y}\quad\fa 0\leq \Theta < 1\fa x,y\in I
        \end{align*}
    \end{definition}

    \begin{definition}[Logarithmische Konvexität]
        Eine Funktion $F$ heißt \emph{logarithmisch konvex}, falls $\log \circ F = \log\of{F}$ konvex ist. Das heißt
        \begin{align*}
            \log F\of{\Theta x + \pair{1-\Theta} y} &\leq \Theta\log F\of{x} + \pair{1-\Theta}\cdot \log F\of{y}\\
            &= \log\of{F\of{x}^{\Theta}} + \log\of{F\of{y}^{1-\Theta}}\\
            &= \log\of{F\of{x}^{\Theta}\cdot F\of{y}^{1-\Theta}}\\
            \equivalent F\of{\Theta x + \pair{1-\Theta}y} &\leq F\of{x}^{\Theta} \cdot F\of{y}^{1-\Theta}\quad\fa x,y\in I\fa 0\leq \Theta \leq 1
        \end{align*}
        dann ist $F$ logarithmisch konvex.
    \end{definition}

    \begin{satz} % Satz 2
        Die $\Gamma$-Funktion $\Gamma: \pair{0, \infty}\fromto \pair{0, \infty}, x\mapsto F\of{x}$ ist logarithmisch konvex.
        \begin{proof}
        (Übung)
        \end{proof}
    \end{satz}

    \begin{satz}[Bohn, Mallerup] % Satz 3
        Ist $F: \pair{0, \infty}\fromto\pair{0, \infty}$ eine Funktion mit
        \begin{enumerate}[label=(\alph*)]
            \item $F\of{1} = 1$
            \item $F\of{x+1} = x\cdot F\of{x}$ und
            \item $F$ ist logarithmisch konvex
        \end{enumerate}
        Dann gilt $F = \Gamma$, das heißt $F\of{x} = \Gamma\of{x}~\forall x > 0$.
        \begin{proof}
            Es reicht zu zeigen, dass die obigen Eigenschaften die Funktion $F$ eindeutig bestimmen, da wir bereits wissen, dass $\Gamma$ die Eigenschaften erfüllt.\\[.5\baselineskip]
            \textsc{Schritt 1}:
            \begin{align*}
                F\of{x+n} \annot[{&}]{=}{(b)} \pair{x+n-1}\cdot F\of{x+n-1}\\
                &= \pair{x+n-1}\cdot\pair{x-n-2}\cdot \ldots \cdot \pair{x-1}\cdot x\cdot F\of{x}\quad\forall x > 0
                \intertext{Für $n\in\N$ gilt}
                F\of{n+1} &= n!\cdot F\of{1} = n!\\
                \impl F\of{n} &= \Gamma\of{n}\quad\forall n\in\N
                \intertext{Das heißt es reicht zu zeigen, dass $F\of{x}$ bei $0< x < 1$ eindeutig bestimmt ist.\endgraf\noindent\textsc{Schritt 2}: Sei $0 < x < 1$}
                s+ n &= \pair{1-x}\cdot n + x\cdot\pair{n+1}\tag{$\Theta = 1-x$}\\
                \annot{\impl}{(c)} F\of{x+n} &= F\of{\pair{1-x}\cot n + x\cdot\pair{n+1}}\\
                &\leq F\of{n}^{1-x} \cdot F\of{n-1}^{x} = F\of{n}^{1-x} \cdot \pair{n\cdot F\of{n}}^x\\
                &= F\of{n}\cdot n^x = \pair{n-1}! \cdot n^x\quad\fa n\in\N\fa 0 < x < 1\\[10pt]
                n+1 &= x\cdot\pair{n+x} + \pair{1-x}\cdot\pair{n+1+x}\tag{1}\\
                \impl F\of{n+x} &\leq F\of{n+x}^x\cdot F\of{n+1+x}^{1-x}\\
                &= F\of{n+x}\cdot \pair{n+x}^{1-x}\tag{2}
                \intertext{Durch die Kombination von (1) und (2) folgt}
                \impl n!\cdot \pair{n+x}^{x-1} &\leq F\of{n+x}\leq \pair{n-1}!\cdot x^n\\
                F\of{n+x} &= x\cdot\pair{x+1}\cdot\ldots\cdot \pair{x+n-1}\cdot F\of{x}\\
                \impl \frac{n!\cdot\pair{n+x}^{x-1}}{x\cdot \pair{x+1}\cdot\ldots\cdot\pair{x+n-1}} &\leq F\of{x} \leq \frac{\pair{n-1}!\cdot n^x}{x\cdot\pair{x+1}\cdot\ldots\cdot \pair{x+n-1}}\\
                a_n\of{x} &\coloneqq \frac{n!\cdot\pair{x+n}^{x-1}}{x\cdot \pair{x+1}\cdot\ldots\cdot\pair{x+n-1}}\\
                b_n\of{x} &\coloneqq \frac{\pair{n-1}!\cdot n^x}{x\cdot\pair{x+1}\cdot\ldots\cdot \pair{x-n-1}}\\
                \impl a_n\of{x} &\leq F\of{x} \leq b_n\of{x}\quad\fa x\in\N\fa 0 < x < 1\\
                \impl \frac{a_n\of{x}}{b_n\of{x}} &\leq \frac{F\of{x}}{b_n\of{x}} \leq 1\\
                \frac{b_n\of{x}}{a_n\of{x}} &= \frac{n^x}{n\cdot\pair{n+x}^{x-1}} = \frac{\pair{n+x}\cdot n^x}{n\cdot\pair{n+x}^x}\\
                &= \frac{n+x}{n}\cdot\pair{\frac{n}{n+x}}^x\underset{\ntoinf}{\fromto}  1\\[10pt]
                \fromto F\of{x} &= \lim_{\ntoinf} b_n\of{x}\\
                &= \lim_{\ntoinf} \frac{\pair{n-1}! \cdot n^x}{x\cdot\pair{1+x}\cdot \ldots\cdot \pair{x+n-1}}\numberthis\label{eq:gamma-alt}
            \end{align*}
            also ist $F\of{x}$ eindeutig bestimmt.
        \end{proof}
    \end{satz}

    \begin{korollar}[Gaußsche Darstellung von $\Gamma$] % Korollar 4
        \begin{align*}
            \Gamma\of{x} &= \lim_{\ntoinf} \frac{n!\cdot x^n}{x\cdot\pair{x+1}\cdot\ldots\cdot \pair{x+n}}\numberthis\label{eq:gamma-gauss}
        \end{align*}
        \begin{proof}
            Da $\frac{n}{n+1}\fromto 1$ für $\ntoinf$ folgt die Behauptung for $0 < x < 1$ direkt aus (\ref{eq:gamma-alt}). Für $x=1$ rechnet sich die Formel leicht nach.
            Also ist noch zu zeigen: Gilt (\ref{eq:gamma-gauss}) für ein $x$, so gilt die Aussage auch für $y=x+1$.
            \begin{align*}
                \Gamma\of{y} &=\Gamma\of{x+1} = x\cdot\Gamma\of{x}\\
                &= x \cdot \lim_{\ntoinf} \frac{n!\cdot n^x}{x\cdot\pair{x+1}\cdot\ldots\cdot\pair{x+n}}\\
                &= \lim_{\ntoinf} \frac{n!\cdot n^{y-1}}{y\cdot\pair{y+1}\cdot\ldots\cdot \pair{y+n-1}}\\
                \intertext{Multiplikation im Zähler mit $n$ und im Nenner mit $y+n$ (was sich für $\ntoinf$ entspricht) liefert}
                &= \lim_{\ntoinf} \frac{n!\cdot n^{y}}{y\cdot\pair{y+1}\cdot \ldots\cdot \pair{y+n-1}\cdot\pair{y+n}}\qedhere
            \end{align*}
        \end{proof}
    \end{korollar}

    \begin{satz} % Satz 5
        \begin{align*}
            \Gamma\of{\frac{1}{2}} &= \sqrt {\pi}
        \end{align*}

        \begin{proof}
            \begin{align*}
                \Gamma\of{\frac{1}{2}} &= \lim_{\ntoinf} \frac{n!\cdot n^{\frac{1}{2}}}{\frac{1}{2}\cdot \pair{1+\frac{1}{2}}\cdot\ldots \cdot \pair{n+\frac{1}{2}}}\\
                &= \lim_{\ntoinf} \frac{n!\cdot n^{\frac{1}{2}}}{\pair{1-\frac{1}{2}}\cdot\pair{2-\frac{1}{2}}\cdot\ldots\cdot \pair{n+1-\frac{1}{2}}}\\
                \impl \Gamma\of{\frac{1}{2}}^{2} &= \lim_{\ntoinf} \frac{2n\cdot \pair{n!}^2}{\pair{1+\frac{1}{2}}\cdot \pair{1-\frac{1}{2}}\cdot\pair{2+\frac{1}{2}}\cdot\pair{2-\frac{1}{2}}\cdot\ldots\cdot \pair{n+\frac{1}{2}}\cdot \pair{n-\frac{1}{2}}}\\
                &= \lim_{\ntoinf} \frac{2n\cdot \pair{n!}^2}{\pair{-\frac{1}{4}}\cdot\pair{4-\frac{1}{4}}\cdot\ldots\cdot\pair{n^2-\frac{1}{4}}}\\
                &= 2\lim_{\ntoinf} \prod_{k=1}^{n} \frac{k^2}{k^2-\frac{1}{2}} = \pi\tag{Wallisches Produkt}
            \end{align*}
        \end{proof}
    \end{satz}

\end{document}
