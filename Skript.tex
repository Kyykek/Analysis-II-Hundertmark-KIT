\documentclass[11pt, twoside, a4paper]{article}

% Setup
\usepackage[margin=2.4cm, top=3.5cm]{geometry}
\usepackage[utf8]{inputenc}
\usepackage[ngerman]{babel}

% Package imports
\usepackage{amsfonts}
\usepackage{amsmath}
\usepackage{amssymb}
\usepackage{amsthm}
\usepackage{mathtools}
\usepackage{setspace}
\usepackage{float}
\usepackage{enumitem}
\usepackage{hyperref}
\usepackage[pagestyles]{titlesec}
\usepackage{fancyhdr}
\usepackage{colonequals}
\usepackage{caption}
\usepackage{tikz}
\usepackage{marginnote}
\usepackage{etoolbox}
\usepackage{mdframed}
\usepackage{aligned-overset}
\usepackage{esint}

% Font-Encoding
\usepackage[T1]{fontenc}
\usepackage{lmodern}

% Theorems
\newtheoremstyle{plain}{}{}{}{}{\bfseries}{.}{ }{}
\theoremstyle{plain}
\newtheorem{blockelement}{Blockelement}[subsection]
\newtheorem{bemerkung}[blockelement]{Bemerkung}
\newtheorem{definition}[blockelement]{Definition}
\newtheorem{lemma}[blockelement]{Lemma}
\newtheorem{satz}[blockelement]{Satz}
\newtheorem{notation}[blockelement]{Notation}
\newtheorem{korollar}[blockelement]{Korollar}
\newtheorem{uebung}[blockelement]{Übung}
\newtheorem{beispiel}[blockelement]{Beispiel}
\newtheorem{folgerung}[blockelement]{Folgerung}
\newtheorem{axiom}[blockelement]{Axiom}
\newtheorem{beobachtung}[blockelement]{Beobachtung}
\newtheorem{konzept}[blockelement]{Konzept}
\newtheorem{visualisierung}[blockelement]{Visualisierung}
\newtheorem{anwendung}[blockelement]{Anwendung}
\newtheorem{skizze}[blockelement]{Skizze}
\newtheorem{genv}[blockelement]{}

% Equation numbering
\numberwithin{equation}{subsection}
\newcommand{\numberthis}[0]{\addtocounter{equation}{1}\tag{\theequation}}

% Marginnotes left
\makeatletter
\patchcmd{\@mn@@@marginnote}{\begingroup}{\begingroup\@twosidefalse}{}{\fail}
\reversemarginpar
\makeatother

% Long equations
\allowdisplaybreaks

% \left \right
\newcommand{\set}[1]{\left\{#1\right\}}
\newcommand{\pair}[1]{\left(#1\right)}
\newcommand{\of}[1]{\mathopen{}\mathclose{}\bgroup\left(#1\aftergroup\egroup\right)}
\newcommand{\abs}[1]{\left\lvert#1\right\rvert}
\newcommand{\norm}[1]{\left\lVert#1\right\rVert}
\newcommand{\linterv}[1]{\left[#1\right)}
\newcommand{\rinterv}[1]{\left(#1\right]}
\newcommand{\interv}[1]{\left[#1\right]}
\newcommand{\sprod}[1]{\left<#1\right>}

% Shorten commands
\newcommand{\equivalent}[0]{\Leftrightarrow{}}
\newcommand{\impl}[0]{\Rightarrow{}}
\newcommand{\fromto}{\rightarrow{}}
\newcommand{\definedas}[0]{\coloneqq}
\newcommand{\definedasbackwards}[0]{\eqqcolon}
\newcommand{\definedasequiv}[0]{\ratio\Leftrightarrow{}}
\newcommand{\exclude}[0]{\setminus}
\renewcommand{\emptyset}{\varnothing}
\newcommand{\sbset}{\subseteq}
\newcommand{\dif}{\mathop{}\!\mathrm{d}}

\newcommand{\ntoinf}[0]{n\fromto\infty}
\newcommand{\toinf}{\fromto\infty}
\newcommand{\fa}{\;\forall\,}
\newcommand{\ex}{\;\exists\,}
\newcommand{\conj}[1]{\overline{#1}}

\newcommand{\annot}[3][]{\overset{\text{#3}}#1{#2}}
\newcommand{\biglim}[1]{{\displaystyle \lim_{#1}}}
\newcommand{\nn}[0]{\\[2\baselineskip]}
\newcommand{\anf}[1]{\glqq{}#1\grqq}
\newcommand{\OBDA}{o.B.d.A. }
\newcommand{\theoremescape}{\leavevmode}
\newcommand{\aligntoright}[2]{\hfill#1\hspace{#2\textwidth}~}
\newcommand{\horizontalline}[0]{\par\noindent\rule{0.05\textwidth}{0.1pt}\\}
\newcommand{\rgbcolor}[3]{rgb,255:red,#1;green,#2;blue,#3}
\newcommand{\fixedspace}[2]{\makebox[#1][l]{#2}}
\newcommand{\ov}[1]{\overline{#1}}
\newcommand{\un}[1]{\underline{#1}}

\let\Re\relax
\let\Im\relax

% MathOperators
\DeclareMathOperator{\grad}{Grad}
\DeclareMathOperator{\bild}{Bild}
\DeclareMathOperator{\Re}{Re}
\DeclareMathOperator{\Im}{Im}
\DeclareMathOperator{\arcsinh}{arcsinh}
\DeclareMathOperator{\arccosh}{arccosh}


% Mengenbezeichner
\newcommand{\R}{\mathbb{R}}
\newcommand{\N}{\mathbb{N}}
\newcommand{\C}{\mathbb{C}}
\newcommand{\Z}{\mathbb{Z}}
\newcommand{\Q}{\mathbb{Q}}
\newcommand{\K}{\mathbb{K}}

\newcommand{\mR}{\mathcal{R}}
\newcommand{\mB}{\mathcal{B}}
\newcommand{\mC}{\mathcal{C}}
\newcommand{\mJ}{\mathcal{J}}
\newcommand{\mPC}{\mathcal{PC}}

\newcommand\imaginarysubsection[1]{
    \refstepcounter{subsection}
    \subsectionmark{#1}
}

% Unfassbar hässlich, aber effektiv für temporäre schnelle Lösungen
\def\:={\coloneqq}
\def\->{\fromto}
\def\=>{\impl}
\def\<={\leq}
\def\>={\geq}

% Envs
\newenvironment{induktionsanfang}{
    \rule{0pt}{3ex}\noindent
    \begin{minipage}[t]{0.11\textwidth}
    {I-Anfang}
    \end{minipage}
    \hfill
    \begin{minipage}[t]{0.89\textwidth}
    }
    {
    \end{minipage}
}
\newenvironment{induktionsvoraussetzung}{
    \rule{0pt}{3ex}\noindent
    \begin{minipage}[t]{0.11\textwidth}
    {I-Vor.}
    \end{minipage}
    \hfill
    \begin{minipage}[t]{0.89\textwidth}
    }
    {
    \end{minipage}
}
\newenvironment{induktionsschritt}{
    \rule{0pt}{3ex}\noindent
    \begin{minipage}[t]{0.11\textwidth}
    {I-Schritt}
    \end{minipage}
    \hfill
    \begin{minipage}[t]{0.89\textwidth}
    }
    {
    \end{minipage}
}

% Section style
\titleformat*{\section}{\LARGE\bfseries}
\titleformat*{\subsection}{\large\bfseries}

% Page styles
\newpagestyle{pagenumberonly}{
    \sethead{}{}{}
    \setfoot[][][\thepage]{\thepage}{}{}
}
\newpagestyle{headfootdefault}{
    \sethead[][][\thesubsection~\textit{\subsectiontitle}]{\thesection~\textit{\sectiontitle}}{}{}
    \setfoot[][][\thepage]{\thepage}{}{}
}
\pagestyle{headfootdefault}

\begin{document}
    \title{\vspace{3cm} Skript zur Vorlesung\\Analysis II\\bei Prof. Dr. Dirk Hundertmark}
    \author{Karlsruher Institut für Technologie}
    \date{Sommersemester 2024}
    \maketitle
    \begin{center}
        Dieses Skript ist inoffiziell. Es besteht kein\\Anspruch auf Vollständigkeit oder Korrektheit.
    \end{center}
    \thispagestyle{empty}
    \newpage

    \tableofcontents
    ~\\
    Alle mit [*] markierten Kapitel sind noch nicht Korrektur gelesen und bedürfen eventuell noch Änderungen.

    \newpage

    \include{Kapitel/Riemann_Integral}
    \include{Kapitel/Orientiertes_Riemann_Integral}
    \include{Kapitel/Hauptsatz_Integral_Differentialrechnung}
    \section{[*] Uneigentliche Integrale}
\thispagestyle{pagenumberonly}
Bisher haben wir immer nur Integrale auf kompakten Intervalle $I$ berechnet und dabei waren alle Funktionen $f\in\mR\of{I}$ insbesondere beschränkt.\\
Frage: Was ist $ \int_{0}^{1} \frac{1}{\sqrt{x}} \dif x$? Was ist $ \int_{0}^{\infty} e^{-t} \dif t$?
\begin{align*}
    \int_{a}^{b} e^{-t} \dif t &= \interv{-e^{-t}}_a^b = e^{-0} - e^{-b} = 1 - e^{-b} = 1 -\frac{1}{e^b}\fromto 1 \text{ für } b\fromto\infty
\end{align*}

\subsection{Uneigentliche Integrale: Fall I}
Es sei $I=\linterv{a, \infty}$, $f: I\fromto\R$ und $f\in\mR\of{\interv{a,b}}~\forall a<b<\infty$ sowie $F\of{b} = \int_{a}^{b} f\of{x} \dif x$.
\begin{definition}[Fall]
    Wir definieren
    \begin{align*}
        \int_{a}^{\infty} f\of{x} \dif x &\coloneqq \lim_{b\toinf} F\of{b} = \lim_{b\toinf} \int_{0}^{b} f\of{x} \dif x
    \end{align*}
    sofern der Grenzwert existiert nennen wir das das uneigentliche Integral von $f$ über $\linterv{a,\infty}$. Wenn der Grenzwert existiert, sagen wir das Integral konvergiert.\\
    Divergiert das Integral und gilt $F\of{b}\toinf$ für $b\toinf$ (oder $F\of{b}\fromto -\infty$ für $b\toinf$), so nennen wir das Integral bestimmt divergent und schreiben
    \begin{align*}
        \int_{a}^{\infty} f\of{x} \dif x &= +\infty
        \intertext{oder}
        \int_{a}^{\infty} f\of{x} \dif x &= -\infty
    \end{align*}
\end{definition}

\begin{satz} % Satz 2
    \label{satz:int-uneigentlich-epsilon}
    Das Integral $ \int_{a}^{\infty} f\of{x} \dif x$ existiert genau dann, wenn
    \begin{align*}
        \forall\varepsilon > 0\ex R\geq a\colon \abs{F\of{b_2} - F\of{b_1}} &= \abs{ \int_{b_1}^{b_2} f\of{x} \dif x} < \varepsilon\quad\forall b_1, b_2 \geq R
    \end{align*}
    \begin{proof}
        Wir wollen die Existenz von $\biglim{b\toinf} F\of{b}$ für $F\of{b} = \int_{a}^{b} f\of{x} \dif x$. Dann folgt der Satz aus dem Cauchy-Kriterium für Grenzwerte.
    \end{proof}
\end{satz}

\begin{definition}[Absolut konvergente uneigentliche Integrale]
    Das Integral
    \begin{align*}
        \int_{a}^{\infty} f\of{x} \dif x
        \intertext{heißt absolut konvergent, falls}
        \int_{a}^{\infty} \abs{f\of{x}} \dif x
    \end{align*}
    konvergiert.
\end{definition}

\begin{satz}
    Ist das Integral $\int_{a}^{\infty} f\of{x} \dif x$ absolut konvergent, so ist es auch konvergent. Das heißt ist $ \int_{a}^{\infty} \abs{f\of{x}} \dif x < \infty$, so konvergiert auch $ \int_{a}^{\infty} f\of{x} \dif x$.
    \begin{proof}
        Wir setzen $G\of{b} = \int_{a}^{b} \abs{f\of{x}} \dif x$ und $F\of{b} = \int_{a}^{b} f\of{x} \dif x$. Wir nehmen an, dass $\biglim{b\toinf} G\of{b}$ existiert, das heißt
        \begin{align*}
            \forall\varepsilon > 0\ex R\geq a\colon \abs{G\of{b_2} - G\of{b_1}} &< \varepsilon\quad\forall b_1, b_2\geq R\\
            \impl \abs{F\of{b_2} - F\of{b_1}} &= \abs{ \int_{b_1}^{b_2} f\of{x} \dif x}\\
            &\leq \int_{b_1}^{b_2} \abs{f\of{x}} \dif x = G\of{b_2} - G\of{b_1}
        \end{align*}
        Damit folgt die Behauptung aus Satz~\ref{satz:int-uneigentlich-epsilon}.
    \end{proof}
\end{satz}

\begin{satz} % Satz 5
    \label{satz:int-majorant}
    Sei $\varphi: \linterv{a,\infty}\fromto\linterv{0, \infty}$ mit
    \begin{align*}
        \int_{a}^{\infty} \varphi\of{x} \dif x &< \infty
        \intertext{une es existiert ein $R_0 \geq 0$, sodass}
        \abs{f\of{x}} &\leq \varphi\of{x}\quad\forall x \geq R_0
        \intertext{Dann ist}
        \int_{a}^{\infty} f\of{x} \dif x
    \end{align*}
    absolut konvergent.
    \begin{proof}
        Für $b_2 \geq b_1\geq R_0$ gilt
        \begin{align*}
            \abs{F\of{b_2} - F\of{b_1}} &= \abs{ \int_{b_1}^{b_2} f\of{x} \dif x}\\
            &\leq \int_{b_1}^{b_2} \abs{f\of{x}} \dif x < \int_{b_1}^{b_2} \varphi\of{x} \dif x\\
            &\leq \int_{b_1}^{b_2} \varphi\of{x} \dif x\fromto 0 \text{ für } b_1\toinf
        \end{align*}
    \end{proof}
\end{satz}

\begin{beispiel}
    Das Integral
    \begin{align*}
        \int_{a}^{\infty} \frac{\sin x}{x} \dif x&
        \intertext{ist konvergent, aber nicht absolut konvergent. Wir definieren}
        f\of{x} &= \begin{cases}
                       \frac{\sin x}{x} &x\neq 0\\
                       1 &x = 0
        \end{cases}
        \intertext{Damit ist $f$ stetig auf $\pair{-\infty, \infty}$ und damit folgt $f\in\mR\of{\interv{a,b}}~\forall a,b\in\R$. Insbesondere existiert}
        \int_{0}^{1} \frac{\sin x}{x} \dif x&\\
        \int_{a}^{b} \frac{\sin x}{x} \dif x &= \int_{a}^{1} \frac{\sin x}{x} \dif x + \int_{1}^{b} \frac{\sin x}{x} \dif x\\
        \int_{1}^{b} \frac{\sin x}{x} \dif x &= \interv{-\cos + \frac{1}{x}}_1^b - \int_{1}^{b} \frac{\cos x}{x^2} \dif x\\
        &= \cos 1 - \frac{\cos b}{b} - \int_{1}^{b} \frac{\cos x}{x^2} \dif x
        \intertext{Wir definieren $\varphi\of{x} = \frac{1}{x^2}$ mit}
        \int_{1}^{b} \frac{1}{x^2} \dif x &= \interv{-\frac{1}{x}}_1^b = 1 - \frac{1}{b}\fromto 1
        \intertext{Außerdem gilt}
        \abs{\frac{\cos x}{x^2}} &\leq \frac{1}{x^2}
        \intertext{Damit ist das Integral nach dem Majorantenkriterium konvergent. Um einzusehen, dass es nicht absolut konvergent ist, betrachten wir für $N\in\N$}
        \int_{N\pi}^{\pair{N+1}\pi} \abs{\frac{\sin x}{\pi}} \dif x &= \int_{N\pi}^{\abs{N+1}\pi} \frac{\abs{\sin x}}{x} \dif x\\
        &\geq \frac{1}{\pi\pair{N+1}} \cdot \int_{N\pi}^{\pair{N+1}\pi} \abs{\sin x} \dif x\\
        \impl \int_{0}^{\pair{k+1}\pi} \abs{\frac{\sin x}{x}} \dif x &= \sum_{n=0}^{k} \int_{n\pi}^{\pair{n+1}\pi} \frac{\abs{\sin x}}{x} \dif x\\
        &\geq \sum_{n=0}^{k} \frac{2}{\pi\pair{n+1}} = \frac{2}{\pi} \sum_{n=0}^{k} \frac{1}{n+1}\fromto \infty
    \end{align*}
\end{beispiel}

\begin{bemerkung}
    Analog zu $\linterv{a, \infty}$ wollen wir auch die Integrale in $\rinterv{-\infty, b}$ betrachten. Wir setzen
    \begin{align*}
        F\of{a} &= \int_{a}^{b} f\of{x} \dif x\\
        \int_{-\infty}^{b} f\of{x} \dif x &\coloneqq \lim_{a\fromto -\infty} \int_{a}^{b} f\of{x} \dif x
    \end{align*}
    sofern der Grenzwert existiert. Alle Aussagen für $\linterv{a, \infty}$ gelten analog auch für $\rinterv{-\infty, b}$.
\end{bemerkung}

\begin{definition} % Definition 6
    Sei $f: \pair{-\infty, \infty}\fromto \R$ und $f\in\mR\of{\interv{a,b}}~\forall a,b\in\R$. Dann nehmen wir $c\in\R$ beliebig und definieren, dass
    \begin{align*}
        \int_{-\infty}^{\infty} f\of{x} \dif x&
        \intertext{konvergiert, falls}
        \int_{-\infty}^{c} f\of{x} \dif& \text{ und } \int_{c}^{\infty} f\of{x} \dif x
        \intertext{beide konvergieren. Und setzen}
        \int_{-\infty}^{\infty} f\of{x} \dif x &\coloneqq \int_{-\infty}^{c} f\of{x} \dif x + \int_{c}^{\infty} f\of{x} \dif x
    \end{align*}
\end{definition}

\begin{uebung}
    Weisen Sie nach, dass sowohl die Konvergenz, als auch der Wert des Integrals in der vorherigen Definition unabhängig von der Wahl von $c$ ist.
\end{uebung}

\begin{bemerkung}
    Es ist allerdings zu beachten, dass
    \begin{align*}
        \lim_{a\toinf} \int_{a}^{c} f\of{x} \dif x + \lim_{b\toinf} \int_{c}^{b}  \dif x &\neq \lim_{R\toinf} \int_{-R}^{R} f\of{x} \dif x
    \end{align*}
    Das heißt die Integrale müssen tatsächlich getrennt betrachtet werden. Zum Beispiel bei der Funktion $f\of{x} = x$ geht $ \int_{-R}^{R} x \dif x \fromto 0$, aber ist eigentlich nicht auf $\pair{-\infty, \infty}$ integrierbar, da sich bei der Trennung in zwei Integrale kein Grenzwert ergibt.
\end{bemerkung}

\subsection{Uneigentliche Integrale: Fall II}
Es sei $I=\linterv{a, b}$ (oder $I=\rinterv{a, b}$) und $f:I\fromto\R$ unbeschränkt bei $x=a$ (oder $x=b$). Außerdem $f\in\mR\of{\interv{a,c}}~\forall a<c < b$ (oder $f\in\mR\of{\interv{c, b}}~\forall a < c < b$)

\begin{definition}
    Existiert
    \begin{align*}
        \lim_{c\fromto b-} \int_{a}^{c} f\of{x} \dif x\quad &\pair{\text{oder } \lim_{c\fromto a+} \int_{c}^{b} f\of{x} \dif x}
        \intertext{so setzen wir}
        \int_{a}^{b} f\of{x} \dif x = \lim_{c\fromto b-} \int_{a}^{c} f\of{x} \dif x\quad &\pair{\text{oder } \int_{a}^{b} f\of{x} \dif x = \lim_{c\fromto a+} \int_{c}^{b} f\of{x} \dif x}
        \intertext{und sagen}
        \int_{a}^{b} f\of{x} \dif x
    \end{align*}
    konvergiert.
\end{definition}

\begin{satz}
    Ist $\abs{f\of{x}} \leq \varphi\of{x}~\forall x\in\linterv{a,b}$ (oder $\forall x\in\rinterv{a,b}$) und konvergiert $ \int_{a}^{b} \varphi\of{x} \dif x$, so konvergiert auch $ \int_{a}^{b} f\of{x} \dif x$
\end{satz}

\begin{beispiel}
    Sei $f: \rinterv{0, 1}\fromto\R,~x\mapsto \frac{1}{\sqrt{x}}$. Dann gilt $F\of{x}  = 2\sqrt{x}$
    \begin{align*}
        \int_{0}^{1} \frac{1}{\sqrt{x}} \dif x &= \interv{2\sqrt{x}}_c^1 = 2-2\sqrt{c}\fromto 2
    \end{align*}
\end{beispiel}

\subsection{Uneigentliche Integrale Fall III}
\marginnote{[14. Mai]}
$f$ hat eine Singularität in $\xi$ im Inneren von $\interv{a,b}$.
\begin{beispiel}
    $f\of{x} = \frac{1}{\abs{\sqrt{x}}}$ auf $\linterv{-1, 0} \cup \rinterv{0, 1}$.
\end{beispiel}

\begin{definition}
    Wir sagen, dass
    \begin{align*}
        \int_{a}^{b} f\of{x} \dif x
        \intertext{existiert/konvergiert, falls die uneigentlichen Integrale}
        \int_{\xi}^{b} f\of{x} \dif x &\text{ und } \int_{a}^{\xi} f\of{x} \dif x
        \intertext{konvergieren. Wir setzen}
        \int_{a}^{b} f\of{x} \dif x &\coloneqq \int_{a}^{\xi} f\of{x} \dif x + \int_{\xi}^{b} f\of{x} \dif x\numberthis\label{eq:un-iii}
    \end{align*}
\end{definition}

\begin{bemerkung}
(\ref{eq:un-iii})
    ist stärker als die Existenz von
    \begin{align*}
        \lim_{\varepsilon\searrow} \int_{I_{\varepsilon}}^{} f\of{x} \dif x
    \end{align*}
    mit $I=\interv{a,b}$ und $I_{\varepsilon}\coloneqq I\exclude\pair{\xi-\varepsilon, \xi+\varepsilon} = \interv{a, \xi-\varepsilon} \cup \interv{\xi+\varepsilon, b}$. (Cauchyscher Hauptwert).
\end{bemerkung}

\begin{beispiel}
    Sei $f\of{x} = \frac{1}{x^2}$, $I=\interv{-1, 1}$. Dann existiert der Cauchysche Hauptwert, aber nicht (\ref{eq:un-iii}).
\end{beispiel}

\subsection{Uneigentliche Integrale Fall IV}

\begin{definition}
    Man hat Singularitäten in $\R$ für $f$ oder/und $b=+\infty$, $a=-\infty$. Dann zerlege $\linterv{a, \infty}$ oder $\rinterv{-\infty, b}$ oder $\pair{-\infty, \infty}$ in endlich viele Intervalle, wobei die Singularitäten die Randpunkte sind (oder $-\infty$, $\infty$). Dann existiert das Integral, falls die endlich vielen uneigentlichen Integrale existieren. Dann nehme Summe aller dieser uneigentlichen Integrale
\end{definition}

\begin{satz}[Integralvergleichskriterium] % Satz 10
    \label{satz:integral-vergleich}
    Sei $f: \linterv{1, \infty} \fromto\R$ monoton fallend. Dann gilt
    \begin{align*}
        \sum_{n=1}^{\infty}  f\of{n} \text{ konvergiert } \equivalent \int_{1}^{\infty} f\of{x} \dif x \text{ existiert }
    \end{align*}
    \begin{proof}
        Siehe Saalübung.
    \end{proof}
\end{satz}

\begin{beispiel}
    Es sei $f\of{x} = x^{-p}$ mit $p\neq 1$. Dann ist $F\of{x} = \frac{1}{1-p}x^{1-p}$ für $F'=f$.
    \begin{align*}
        \int_{1}^{\infty} \frac{1}{x^p} \dif x &= \lim_{R\toinf} \interv{\frac{1}{1-p}x^{1-p}}_1^R
    \end{align*}
    existiert nach Satz~\ref{satz:integral-vergleich} für $p>1$.
\end{beispiel}

\begin{beispiel}
    $f\of{x} = \log_2\of{x} = \log\of{\log\of{x}}$, $x > 1$
    \begin{align*}
        \frac{\dif}{\dif x} \log_2\of{x} &= \frac{1}{\log\of{x}}\cdot\frac{1}{x}\\
        \frac{\dif}{\dif x}\pair{\log_2\of{x}}^{1-s} &= \frac{1-s}{\pair{\log x}^s}\cdot \frac{1}{x}\\
        \impl \sum_{n=2}^{\infty} \frac{1}{n\pair{\log^s n}^s} \text{ konvergiert } &\equivalent s > 1
    \end{align*}
\end{beispiel}

\begin{beispiel}[Gamma-Funktion]
    \begin{align*}
        \Gamma\of{x} &\coloneqq \int_{0}^{\infty} t^{x-1} e^{-t} \dif t\tag{$x > 0$}
    \end{align*}
    \begin{enumerate}[label=(\alph*)]
        \item
        \begin{align*}
            t^{x-1}e^{-t} &\leq t^{x-1}\quad\forall t > 0
            \intertext{\item~}
            t^{x-1}e^{-t} &= t^{x-1}e^{-\frac{t}{2}}e^{-\frac{t}{2}}\\
            &\leq c_x e^{-\frac{t}{2}}\quad\forall t\geq 1\tag{$c_x\coloneqq \sup_{t\geq 1} t^{x-1}e^{-\frac{t}{2}}$}
            \intertext{$t^{x-1}e^{-\frac{t}{2}}$ ist beschränkt auf $\linterv{1, \infty}$}
            \int_{0}^{1} t^{x-1}e^{-t} \dif t &\leq \int_{0}^{1} t^{x-1} \dif t\\
            &= \lim_{c\toinf} \interv{\frac{1}{x}t^{x}}_c^1\\
            &= \lim_{c\fromto 0^{+}} \frac{1}{x}\pair{1-e^{x}}\\
            0 &\leq \int_{1}^{\infty} t^{x-1}e^{-t} \dif t\\
            &= \lim_{b\toinf} \int_{a}^{b} t^{x-1}e^{-t} \dif t\\
            &\leq c_x e^{-\frac{t}{2}}\\
            &\leq \lim_{b\toinf} c_x \int_{a}^{b} e^{-\frac{t}{2}} \dif t < \infty\\
            \int_{a}^{b} e^{-\frac{t}{2}} &= \interv{-2e^{-\frac{t}{2}}}_1^b = 2\pair{e^{-\frac{1}{2}} - e^{-\frac{b}{2}}}\fromto 2e^{-\frac{1}{2}}
        \end{align*}
    \end{enumerate}
\end{beispiel}

\begin{satz}[Funktionalgleichung der $\Gamma$-Funktion] % Satz 12
    Es gilt $\Gamma\of{n+1} = n!$ und $x\Gamma\of{x} = \Gamma\of{x+1}$ für alle $x>0$.
    \begin{proof}
        \begin{align*}
            \Gamma\of{x+1} &= \int_{0}^{\infty} t^{(x+1)-1}e^{-t} \dif t\\
            &= \int_{0}^{\infty} t^{x}e^{-t} \dif t\\
            \intertext{Wir integrieren partiell. Sei $0 < a < b < \infty$}
            \int_{a}^{b} t^{x}e^{-t} \dif t &= \interv{-t^x e^{-t}}_a^b + \int_{a}^{b} xt^{x-1}e^{-t} \dif t\\
            &= a^{x}e^{-b} - b^{x}e^{-b} + x \int_{a}^{b} t^{x-1}e^{-t} \dif t\\
            \impl \int_{a}^{\infty} t^x e^{-t} \dif t &= \lim_{b\toinf} \int_{a}^{b} t^x e^{-t} \dif t\\
            &= a^x e^{-a} + x \int_{a}^{\infty} t^{x-1} e^{-t} \dif t\\
            \impl \Gamma\of{x+1} &= \int_{0}^{\infty} t^x e^{-t} \dif x = x\Gamma\of{x}
            \intertext{Damit folgt die zweite Behauptung. Wir betrachten außerdem}
            \Gamma\of{n+1} &= n\Gamma\of{n} = n\Gamma\of{n-1+1}\\
            &= n\pair{n-1}\Gamma\of{n-1} = n\cdot \pair{n-1}\cdot\ldots\cdot 2 \cdot 1 \cdot \Gamma\of{1}\\
            &= n!\qedhere
        \end{align*}
    \end{proof}
\end{satz}

\begin{anwendung}
    Nach Substitution mit $t^2 = x$ gilt $\frac{\dif t}{\dif x} = \frac{1}{2\sqrt{x}}$
    \begin{align*}
        \int_{a}^{\xi} e^{-t^2} \dif t &= \int_{}^{} e^{-x}\frac{1}{2}\sqrt{x} \dif x\\
        &= \frac{1}{2} \int_{0}^{\infty} \frac{1}{\sqrt{x}}e^{-x} \dif x\\
        &= \frac{1}{2} \int_{?}^{b} s^{-\frac{1}{2}} e^{-s} \dif s\\
        \intertext{für $b\toinf$ und $a\searrow 0$}
        \impl 2 \int_{0}^{\infty} e^{-t^2} \dif x &= \int_{0}^{\infty} s^{-\frac{1}{2}} e^{-s} \dif s\\
        &= \Gamma\of{\frac{1}{2}}
    \end{align*}
    Berechnung von $\Gamma\of{\frac{1}{2}}$ später.
\end{anwendung}

\newpage



    \section{[*] Integrale und gleichmäßige Konvergenz}
    \imaginarysubsection{Gleichmäßige Konvergenz}
    \thispagestyle{pagenumberonly}

    $I=\interv{a,b}$ und $f: I\fromto\R$, $f_n: I\fromto \R$. $f_n$ konvergiert gegen $f$ \anf{irgendwie}. Wann gilt
    \begin{align*}
        \int_{a}^{b} f_n\of{x} \dif x \fromto \int_{a}^{b} f\of{x} \dif x
    \end{align*}
    Punktweise Konvergenz ist nicht genug.
    \begin{beispiel}
        Sei $f_n: \interv{0, 1}\fromto\R$ mit
        \begin{align*}
            f_n\of{x} &\coloneqq \begin{cases}
                                     n &0 < x < \frac{1}{n}\\
                                     0 &x = 0 \text{ oder } \frac{1}{n} \leq x \leq 1
            \end{cases}\\
            \int_{0}^{1} f_n\of{x} \dif x &= \int_{0}^{\frac{1}{n}} n \dif x = \frac{n}{n} = 1
        \end{align*}
        das heißt $f_n\of{x}\fromto 0$ für $n\toinf~\forall x\in\interv{0,1}$.
    \end{beispiel}

    \begin{satz} % Satz 1
        \label{satz:gleichm-int}
        Seien $f, f_n: \interv{a,b}\fromto\R \pair{\C, \dots}$ und $n\in\N$. $(f_n)_n$ konvergiere gleichmäßig gegen $f$ auf $\interv{a,b}$ und $f_n\in\mR\of{\interv{a,b}}$. Dann gilt $f\in\mR\of{I}$ und
        \begin{align*}
            \lim_{\ntoinf} \int_{a}^{b} f_n\of{x} \dif x &= \int_{a}^{b} f\of{x} \dif x = \int_{a}^{b} \lim_{\ntoinf} f_n\of{x} \dif x
        \end{align*}

        \begin{proof}
            Sei $\varepsilon > 0$ und $N\in\N$. Dann gilt
            \begin{align*}
                \norm{f-f_n}_{\infty} &= \sup_{a < x < b} \abs{f\of{x} - f_n\of{x}} < \frac{\varepsilon}{4\pair{b-a}}\\
                \impl f_n\of{x} - \frac{\varepsilon}{4\pair{b-a}} &\leq f\of{x} \leq f_n\of{x} + \frac{\varepsilon}{4\pair{b-a}}\quad\forall n\geq N
                \intertext{Halte $N$ fest und nehme Zerlegung $Z$ von $I=\interv{a,b}$ mit}
                \ov{S}_Z\of{f_N} - \un{S}_Z\of{f_N} &< \frac{\varepsilon}{2}\\
                \ov{S}_Z\of{f_N - \frac{\varepsilon}{4\pair{b-a}}} &\leq \ov{S}_Z\of{f}\\
                \ov{S}_Z\of{f_N - \frac{\varepsilon}{4\pair{b-a}}} &= \ov{S}_Z\of{f_N} + \ov{S}_Z\of{\frac{\varepsilon}{4\pair{b-a}}}\\
                &= \ov{S}_Z\of{f_N} + \frac{\varepsilon}{4}\\[5pt]
                \un{S}_Z\of{f_N - \frac{\varepsilon}{4\pair{b-a}}} &\leq \un{S}_Z\of{f}\\
                \un{S}_Z\of{f_N - \frac{\varepsilon}{4\pair{b-a}}} &= \un{S}_Z\of{f_N} - \frac{\varepsilon}{4}\\
                \impl \ov{S}_Z\of{f} - \un{S}_Z\of{f} &\leq \ov{S}_Z\of{f_N} + \frac{\varepsilon}{4} - \pair{\un{S}_Z\of{f_N} - \frac{\varepsilon}{4}}\\
                &= \ov{S}_Z\of{f_N} - \un{S}_Z\of{f_N} + \frac{\varepsilon}{2} < \frac{\varepsilon}{2} + \frac{\varepsilon}{2} = \varepsilon
            \end{align*}
            Damit folgt $f\in\mR\of{I}$.
            \begin{align*}
                \int_{a}^{b} f_n\of{x} \dif x - \frac{\varepsilon}{4} &= \int_{a}^{b} \pair{f_n\of{x} - \frac{\varepsilon}{4\pair{b-a}}} \dif x\\
                &= \int_{a}^{b} f\of{x} \dif x \leq \int_{a}^{b} \pair{f_n\of{x} + \frac{\varepsilon}{4\pair{b-a}}} \dif x\\
                &= \int_{a}^{b} f_n\of{x} \dif x + \frac{\varepsilon}{4}\\
                \impl \limsup_{\ntoinf} \int_{a}^{b} f_n\of{x} \dif x - \frac{\varepsilon}{4} &\leq \int_{a}^{b} f\of{x} \dif x \leq \liminf_{\ntoinf} \int_{a}^{b} f_n\of{x} \dif x + \frac{\varepsilon}{4}\quad\forall\varepsilon > 0\\
                \impl \limsup_{\ntoinf} \int_{a}^{b} f_n\of{x} \dif x &\leq \int_{a}^{b} f\of{x} \dif x \leq \liminf \int_{a}^{b} f_n\of{x} \dif x\qedhere
            \end{align*}
        \end{proof}
    \end{satz}

    \begin{beispiel}[Integral von Potenzreihen]
        \marginnote{[17. Mai]}
        Wir betrachten die Potenzreihe
        \begin{align*}
            f\of{x} &= \sum_{n=0}^{\infty} a_n\pair{x-x_0}^n
            \intertext{mit Konvergenzradius $R>0$ und}
            R &= \frac{1}{\displaystyle \limsup_{\ntoinf} \abs{a_n}^{\frac{1}{n}}}
            \intertext{Wir erhalten also eine Funktion $f: \pair{x_0 - R, x_0 + R}\fromto\R$ (oder $\C$). Die Stammfunktion zu $a_n\pair{x-x_0}^n$ ist $\frac{a_n}{n+1}\pair{x-x_0}^{n+1}$. Wir definieren also eine Funktion $F$ analog}
            F\of{x} = \sum_{n=0}^{\infty} \frac{a_n}{n+1}\pair{x-x_0}^{n+1} &= \sum_{n=1}^{\infty} c_n\pair{x-x_0}^n\tag{$c_n\coloneqq \frac{a_{n-1}}{n}$}\\
            \limsup_{\ntoinf} \pair{\abs{c_n}}^{\frac{1}{n}} &= \limsup_{\ntoinf} \abs{\frac{a_{n-1}}{n}}^{\frac{1}{n}}\\
            \intertext{Es gilt}
            \pair{\frac{\abs{a_{n-1}}}{n}}^{\frac{1}{n}} &= \frac{1}{n^{\frac{1}{n}}}\pair{\abs{a_{n-1}}^{\frac{1}{n-1}}}^{\frac{n-1}{n}}\\
            \impl \limsup_{\ntoinf} \abs{c_n}^{\frac{1}{n}} &= \limsup_{\ntoinf} \abs{a_n}^{\frac{1}{n}}
            \intertext{Das heißt $F$ hat denselben Konvergenzradius wie $f$. Unsere Hoffnung ist also, dass $F$ eine Stammfunktion von $f$ ist oder}
            \int_{x_0}^{x} f\of{t} \dif t &= F\of{x}
            \intertext{Das gilt tatsächlich und lässt sich folgendermaßen zeigen. Wir definieren eine Funktionenfolge}
            f_n\of{x} &= \sum_{k=0}^{n} a_k\pair{x-x_0}^k
            \intertext{Wir wissen $\forall\delta > 0$ klein genug (konkret heißt das $\delta < R$) konvergiert $f_n$ gleichmäßig gegen $f$ auf dem Intervall $\interv{x_0-R+\delta, x_0+R-\delta}$. Dann gilt nach Satz~\ref{satz:gleichm-int} für $x\in\interv{x_0-R+\delta, x_0+R-\delta}$ fest}
            \int_{x_0}^{x} f\of{t} \dif t &= \lim_{\ntoinf} \int_{x_0}^{x} f_n\of{t} \dif t\\
            &= \lim_{\ntoinf} \int_{x_0}^{x} \sum_{k=0}^{n} \frac{a_k}{k+1}\pair{x-x_0}^{k+1} \dif x = F(x)\\
            \int_{x_0}^{x} f_n\of{t} \dif t &= \int_{x_0}^{x} \sum_{k=0}^{n} a_k\pair{x-x_0}^k \dif t = \sum_{k=0}^{n} a_k \int_{x_0}^{x} \pair{t-x_0}^k \dif t\\
            &= \interv{\frac{1}{k+1}\pair{t-x_0}^{k-1}}_{x_0}^{x} = \frac{1}{k+1}x{k+1}
        \end{align*}
    \end{beispiel}

    \begin{satz}
        Sei $I=\interv{a,b}$ sowie $f_n: I\fromto\R$ (oder $\C$) und die folgenden Voraussetzungen gelten
        \begin{enumerate}[label=(\roman*)]
            \item $\exists x_0\in I\colon f_n\of{x_0}$ konvergiert gegen $f\of{x_0}$
            \item $\pair{f_n'}_n$ konvergiert gleichmäßig gegen eine Funktion $g$
            \item $f_n'$ ist stetig für alle $n\in\N$
        \end{enumerate}
        Dann gilt $f(x) \coloneqq \displaystyle\lim_{\ntoinf} f_n\of{x}~\forall x\in I$ und $f$ ist stetig differenzierbar mit Ableitung $f' = g$.
        \begin{proof}
            Sei $x\in I$. Da alle Ableitungen von $f_n$ stetig sind, können wir den Hauptsatz verwenden und es gilt
            \begin{align*}
                f_n\of{x} - f_n\of{x_0} &= \int_{x_0}^{x} f_n'\of{t} \dif t\\
                \impl f_n\of{x} &= \underbrace{f_n\of{x_0}}_{\fromto f\of{x_0}} + \underbrace{\int_{x_0}^{x} f_n'\of{t} \dif t}_{\fromto \int_{x_0}^{x} g\of{t} \dif t}\\
                \impl f\of{x} &\coloneqq \lim_{\ntoinf} f_n\of{x} \text{ existiert } \forall x\in I \text{ und}\\
                f\of{x} &= f\of{x_0} + \int_{x_0}^{x} g\of{t} \dif t
            \end{align*}
            Nach dem Hauptsatz gilt, dass $f$ stetig differenzierbar ist mit $f' = g$.
        \end{proof}
    \end{satz}

    \begin{anwendung}
        \label{anwendung:potenzreihe-diff}
        \begin{align*}
            f\of{x} &= \sum_{n=0}^{\infty} a_N\pair{x-x_0}^{n}\\
            R &= \frac{1}{\displaystyle \limsup_{\ntoinf} \abs{a_n}^{\frac{1}{n}}} > 0\\
            f_n\of{x} &= \sum_{k=0}^{n} a_k\pair{x-x_0}^k\\
            \impl f\of{x} &= \lim_{\ntoinf} f_n\of{x}\\
            f_n'\of{x} &= \sum_{k=1}^{\infty} k\cdot a_k\pair{x-x_0}^{k-1}
            \intertext{Es gilt}
            \limsup_{\ntoinf} \abs{\pair{n+1} a_{n+1}}^{\frac{1}{n}} &= \limsup_{\ntoinf} \abs{a_{n+1}}^{\frac{1}{n+1}}
            \intertext{Nach dem vorherigen Satz gilt damit}
            f_n'\of{x} &= \sum_{k=1}^{n} k\cdot a_k\pair{x-x_0}^{k-1}
            \intertext{konvergiert auch auf $\pair{x_0-R, x_0+R}$ und gleichmäßig auf $\interv{x_0-R+\delta, x_0+R-\delta}$. Also konvergiert}
            \sum_{n=0}^{\infty} a_n\pair{x-x_0}^n &= \of{x}
            \intertext{und ihre Ableitung ist gegeben durch}
            \sum_{n=1}^{\infty} n\cdot a_n\pair{x-x_0}^{n-1}
        \end{align*}
        Also ist jede Potenzreihe differenzierbar auf ihrem Konvergenzintervall.
    \end{anwendung}

    \begin{korollar}
        Jede Potenzreihe ist unendlich oft differenzierbar auf ihrem Konvergenzintervall.
        \begin{proof}
            Nach Anwendung~\ref{anwendung:potenzreihe-diff} ist eine Potenzreihe einmal differenzierbar mit einer Potenzreihe als Ableitung. Damit folgt induktiv die Behauptung.
        \end{proof}
    \end{korollar}

    \begin{beispiel}
        Wir wissen
        \begin{align*}
            \sum_{n=0}^{\infty} x^{n} &= \frac{1}{1-x}\tag{$\abs{x} < 1$}\\
            \impl \sum_{n=1}^{\infty} n\cdot x^{n} &= x\cdot \sum_{n=1}^{\infty} n\cdot x^{n-1}\\
            &= x\cdot \frac{\dif}{\dif x}\cdot \frac{1}{1-x}\\
            &= x\cdot \frac{\dif}{\dif x} \sum_{n=0}^{\infty} x^{n} = x\cdot \frac{-1}{\pair{1-x}^2}\pair{-1} = \frac{x}{\pair{1-x}^2}
        \end{align*}
    \end{beispiel}

    \begin{bemerkung}[Taylorrreihe]
        \begin{align*}
            f\of{x} - f\of{x_0} &= \int_{x_0}^{x} f'\of{t} \dif t\\
            \impl f\of{x} &= f\of{x_0} + \int_{x_0}^{x} f'\of{t} \dif t\\
            &= f\of{x_0} + \int_{x_0}^{x} \pair{f'\of{t} - f'\of{x_0} + f'\of{x_0}} \dif x\\
            &= f\of{x_0} + \int_{x_0}^{x} \pair{f'\of{t} - f'\of{x_0}} \dif t + f'\of{x_0} \cdot \int_{x_0}^{x} 1 \dif t\\
            &= f\of{x_0} + f'\of{x_0}\cdot \pair{x-x_0} + \underbrace{\int_{x_0}^{x} \pair{f'\of{t} - f'\of{x_0}} \dif t}_{\eqqcolon R_{x_0}\of{x}}
            \intertext{Wir können den Fehler abschätzen und erhalten für ein $\varepsilon\of{x} \coloneqq \sup_{t\in\pair{x_0, x}} \abs{f'\of{t} - f'\of{x_0}}$}
            \abs{R_{x_0}\of{x}} &\leq \int_{x_0}^{x} \abs{f'\of{t} - f'\of{x_0}} \dif t \leq \varepsilon\of{x}\cdot\abs{x-x_0}\\
            \frac{\abs{R_{x_0}\of{x}}}{\abs{x-x_0}} &= \varepsilon\of{x} \fromto 0 \text{ für } x\fromto x_0
        \end{align*}
    \end{bemerkung}

\end{document}
