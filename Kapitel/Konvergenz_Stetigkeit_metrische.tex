\imaginarysubsection{[*]Konvergenz und Stetigkeit in metrischen Räumen}
\thispagestyle{pagenumberonly}

\begin{definition}[Konvergenz in metrischen Räumen]
    Sei $\pair{M, d}$ ein metrischer Raum. Eine Folge $(x_n)_n \subseteq M$ konvergiert gegen $a \in M$, falls
    \begin{align*}
        &\forall\varepsilon > 0\ex N\in\N\colon d\of{x_n, a} < \varepsilon\quad\forall n\geq N\\
        (\equivalent &\forall \varepsilon > 0\colon x_n \in B_{\varepsilon}\of{a} \text{ für fast alle } n)
    \end{align*}
    Wir schreiben $a = \lim_{\ntoinf} x_n$ oder $x_n\fromto a$ für $\ntoinf$.\\
    $(x_n)_n$ heißt \emph{Cauchy-Folge}, falls
    \begin{align*}
        &\forall\varepsilon>0\ex N\in\N\colon d\of{x_n, x_m} < \varepsilon\quad\forall n,m\geq N
    \end{align*}
\end{definition}

\begin{definition}[Durchmesser]
    Sei $\pair{M, d}$ ein metrischer Raum. Wir definieren den Durchmesser von einer Menge $A$ mit
    \begin{align*}
        \diam\of{A} &\coloneqq \sup \set{d\of{x,y}: x,y\in A}
    \end{align*}
    $A$ ist beschränkt, falls $\diam\of{A} < \infty$.
\end{definition}

\begin{bemerkung}
    Eine Menge $A\subseteq M$ ist genau dann beschränkt, wenn
    \begin{align*}
        \exists x\in M, r> 0\colon A \subseteq B_r\of{x}
    \end{align*}
    In diesem Fall ist $\diam\of{A} \leq 2r$.
\end{bemerkung}

\begin{satz}
    \theoremescape
    \begin{enumerate}[label=(\roman*)]
        \item Konvergiert die Folge $(x_n)_n$, so ist sie eine Cauchy-Folge
        \item Jede Cauchy-Folge ist beschränkt.
    \end{enumerate}
    \begin{proof}
        ~\\
        \begin{enumerate}[label=(\roman*)]
            \item Sei $a = \lim_{\ntoinf} x_n$ und $\varepsilon > 0$
            \begin{align*}
                \impl \exists N\in\N\colon d\of{x_n , a} &< \frac{\varepsilon}{2}\\
                \impl d\of{x_n, x_m} &\leq d\of{x_n, a} + d\of{a, x_m} < \varepsilon\quad\forall n,m\geq N\\
                \impl (x_n)_n &\text{ ist eine Cauchy-Folge }
            \end{align*}
            \item Sei $(x_n)_n$ eine Cauchy-Folge. Wir wählen $\varepsilon = 1$
            \begin{align*}
                \impl \exists N\in\N\colon d\of{x_n, x_m} &< 1\quad\forall n,m\geq N
                \intertext{Wir definieren $a\coloneqq x_N$}
                \impl d\of{x_n, a} &< 1\quad\forall n\geq N
                \intertext{$r\coloneqq \max\of{d\of{x_1, a}, d\of{x_2, a}, \ldots, d\of{x_{N-1}, a}} + 1$}
                \impl \forall n\in\N\colon d\of{x_n, a} &< r\\
                \impl x_n&\in B_r\of{a}\quad\forall n\in \N
                \intertext{Das heißt}
                \bigcup_{n\in\N} \set{x_n} &\subseteq B_r\of{A} \text{ ist eine beschränkte Menge }
            \end{align*}
        \end{enumerate}
    \end{proof}
\end{satz}

\begin{definition}[Banachräume]
    Ein metrischer Raum $\pair{M, d}$ heißt vollständig, falls jede Cauchy-Folge in $M$ konvergiert. Ein vollständiger, normierter Vektorraum heißt Banachraum.
\end{definition}

\begin{beispiel}
    $\pair{\R^d, \norm{\cdot}_2}$, $\pair{\R^d, \norm{\cdot}_{\infty}}$ oder $\pair{\R^d, \norm{\cdot}_1}$ sind vollständige, normierte Vektorräume. Genauso $\pair{\C^d, \norm{\cdot}_2}$ etc.
\end{beispiel}

\begin{satz}[Schachtelungsprinzip] % Satz 4
    Sei $\pair{M, d}$ ein vollständiger metrischer Raum und $(A_n)_n$ eine absteigende Folge abgeschlossener Mengen ($A_0 \supseteq A_1 \supseteq A_2 \supseteq \dots$) mit $\diam\of{A_n}\fromto 0$ für $\ntoinf$ und $A_n\neq \emptyset~\forall n\in\N$. Dann existiert genau ein $x\in M$ sodass $x\in \bigcap_{n\in\N} A_n$. Das heißt $x\in A_n~\forall n\in\N$.
    \begin{proof}
        \textsc{Eindeutigkeit}: Angenommen $x, y\in A_n~\forall n\in \N$. Dann gilt $d\of{x,y} \leq \diam\of{A_n}\fromto 0$ für $\ntoinf$. Das heißt $d\of{x,y} = 0\equivalent x = y$.\\[.2\baselineskip]
        \textsc{Existenz}: $\forall n\in\N\ex x_n\in A_n$. Behauptung: $(x_n)_n$ ist eine Cauchy-Folge. Sei $n\geq m$
        \begin{align*}
            x_n\in A_n \subseteq A_{n-1} &\subseteq A_{n-2} \subseteq \dots \subseteq A_m
            \intertext{Das heißt}
            x_n &\in A_m\quad\forall n\geq m\\
            \impl x_n, x_m &\in A_m\quad\forall n\geq m\\
            d\of{x_n, x_m} &\leq \diam\of{A_m}\fromto 0 \text{ für } m\toinf\\
            \intertext{Das heißt $(x_n)_n$ ist eine Cauchy-Folge. Nach der Vollständigkeit von $M$ existiert ein $x\coloneqq \lim_{\ntoinf} x_n$. Behauptung: $x\in A_m~\forall m\in\N$}
            x_n &\in A_m\quad\forall n\geq m\\
            \impl \lim_{\ntoinf} x_n &\in A_m \text{ da $A_m$ abgeschlossen ist }\\
            \impl x&\in A_m\quad\forall m\in \N\qedhere
        \end{align*}
    \end{proof}
\end{satz}

\begin{definition}[Stetigkeit in metrischen Räumen]
    \marginnote{[21. Jun]}
    \label{definition:stetigkeit-metr}
    Seien $\pair{M, d_M}$, $\pair{N, d_N}$ metrische Räume.
    \begin{enumerate}[label=(\alph*)]
        \item $\varepsilon$-$\delta$-Definition von Stetigkeit: Eine Funktion $f: M\fromto N$ ist stetig im Punkt $a\in M$ falls
        \begin{align*}
            \forall\varepsilon > 0\ex\delta > 0\colon d_N\of{f\of{x}, f\of{a}} < \varepsilon\quad\forall x\in M \text{ mit } d_M\of{x,a} < \delta
        \end{align*}
        \item Folgenstetigkeit: Eine Funktion $f: M\fromto N$ ist folgenstetig in $a\in M$, falls für jede Folge $(x_n)_n \subseteq M$ mit $\biglim{\ntoinf} x_n = a$ auch $\biglim{\ntoinf} f\of{x_n} = f\of{a}$ folgt.
    \end{enumerate}
\end{definition}

\begin{bemerkung}
    Definition~\ref{definition:stetigkeit-metr} (a) ist dabei äquivalent zu
    \begin{align*}
        \forall \varepsilon > 0\ex \delta > 0\colon f\of{B_{\delta}^M\of{a}} \subseteq B_{\varepsilon}^{N}\of{f\of{a}}
    \end{align*}
\end{bemerkung}

\begin{satz} % Satz 6
    \label{satz:stetigkeit-def-equiv}
    Für eine Funktion $f: M\fromto N$ und $a\in M$ sind äquivalent
    \begin{enumerate}[label=(\roman*)]
        \item $f$ ist $\varepsilon$-$\delta$-stetig in $a$
        \item $f$ ist folgenstetig in $a$
        \item Für jede Umgebung $U$ von $f\of{a}$ ist $V=f^{-1}\of{U}$ eine Umgebung von $a$
    \end{enumerate}
    Dabei ist (iii) die topologische Definition von Stetigkeit.
    \begin{proof}
    (i)
        $\impl$ (ii): Sei $(x_n)_n \subseteq M$ mit $x_n\fromto a$. Für $\varepsilon > 0$ wähle $\delta > 0$, sodass
        \begin{align*}
            d_N\of{f\of{x}, f\of{a}} < \varepsilon\quad\forall x&\in M \text{ mit } d_M\of{x,a} < \delta
            \intertext{Da $a=\biglim{\ntoinf} x_n$ existiert ein $N\in\N$, sodass $\forall n > N\colon d_M\of{x_n, a} < \delta$}
            \impl \forall n > N\colon d_N\of{f\of{x_n}, f\of{a}} &< \varepsilon\\
            \impl f\of{x_n} \fromto f\of{a}&
        \end{align*}
        (ii) $\impl$ (i): Kontraposition. Angenommen (i) gilt nicht. Dann $\exists\varepsilon_0$, sodass
        \begin{align*}
            \forall\delta > 0\ex &x\in M\colon d_M\of{x, a} < \delta \text{ und } d_N\of{f\of{x}, f\of{a}} > \varepsilon_0
            \intertext{Wir wählen $\delta_n = \frac{1}{n}$. Dann gilt $x_n\fromto a$, da $d\of{x_n, a} < \frac{1}{n}$. Aber}
            \exists x_1\colon d_N&\of{f\of{x_1}, f\of{a}} > \varepsilon_0\\
            \exists x_2\colon d_N&\of{f\of{x_2}, f\of{a}} > \varepsilon_0\\
            &\vdots\\
            \exists x_n\colon d_N&\of{f\of{x_n}, f\of{a}} > \varepsilon_0\\
            \impl d_N&\of{f\of{x_n}, f\of{a}}\quad\forall n\in\N > \varepsilon_0\\
            \impl f&\of{x_n} \not\fromto f\of{a}
        \end{align*}
        (i) $\impl$ (iii): Sei $U\subseteq N$ eine Umgebung von $f\of{a}$. Dann gilt
        \begin{align*}
            \exists\varepsilon > 0\colon B_{\varepsilon}^{N}\of{f\of{a}} &\subseteq N\\
            \annot{\impl}{(i)} \exists \delta > 0\colon f\of{B_{\delta}\of{a}} &\subseteq B_{\varepsilon}\of{f\of{a}}\\
            \impl \exists \delta > 0\colon B_{\delta}\of{a} &\subseteq f^{-1}\of{U}\\
            \impl f^{-1}\of{U} \coloneqq V &\text{ ist eine Umgebung }
        \end{align*}
        (iii) $\impl$ (i): Sei $\varepsilon > 0$. Dann ist $B_{\varepsilon}^{N}\of{f\of{a}}$ eine Umgebung von $f\of{a}$. Nach (iii) ist $V\coloneqq f^{-1}\of{B_{\varepsilon}^{N}\of{f\of{a}}}$ eine Umgebung von $a$. Das heißt
        \begin{align*}
            \exists\delta > 0\colon B_{\delta}^{M}\of{a} &\subseteq V\\
            \equivalent d_N\of{f\of{x}, f\of{a}} &< \varepsilon\quad\forall x\in B_{\delta}\of{a}\\
            \equivalent d_N\of{f\of{x}, f\of{a}} &< \varepsilon\quad\forall x\in M \text{ mit } d_M\of{x,a} < \delta\qedhere
        \end{align*}
    \end{proof}
\end{satz}

\begin{definition} % Definition 7
    \label{definition:stetigkeit-vollst-metr}
    Seien $\pair{M, d_M}$ und $\pair{N, d_N}$ metrische Räume und $f: M\fromto N$ eine Funktion.
    \begin{enumerate}[label=(\alph*)]
        \item $\varepsilon$-$\delta$-Version: $f$ heißt stetig auf $M$, falls
        \begin{align*}
            \forall x_0\in M\fa \varepsilon > 0\ex \delta > 0\colon d_N\of{f\of{x}, f\of{x_0}} < \varepsilon\quad\forall x\in M \text{ mit } d_M\of{x, x_0} < \delta
        \end{align*}
        \item Folgenversion: $f$ heißt stetig auf $M$, falls $\forall x_0\in M$ und jede Folge $(x_n)_n$ mit $x_n\fromto x_0$ gilt $f\of{x_n}\fromto f\of{x_0}$.
        \item Topologische Version: $f$ heißt stetig auf $M$, falls für jede offene Menge $U\subseteq N$ das Urbild $f^{-1}\of{U}$ offen in $M$ ist.
    \end{enumerate}
\end{definition}

\begin{satz} % Satz 8
    Für metrische Räume $\pair{M, d_M}$ und $\pair{N, d_N}$ und eine Funktion $f: M\fromto N$ sind die Versionen (a), (b) und (c) von Definition~\ref{definition:stetigkeit-vollst-metr} äquivalent.
    \begin{proof}
        Der Beweis von (a) $\equivalent$ (b) folgt direkt aus Satz~\ref{satz:stetigkeit-def-equiv}.\\
        (c) $\impl$ (a): Sei $U$ offen in $N$. Dann ist $f^{-1}\of{U}$ offen in $M$. Sei $x_0\in M$ beliebig und $\varepsilon > 0$. Dann ist
        \begin{align*}
            B_{\varepsilon}^{N}&\coloneqq B_{\varepsilon}^{N}\of{f\of{x_0}} = \set{y\in N: d_N\of{y, f\of{x_0}} < \varepsilon}
            \intertext{offen in $N$}
            &\impl V=f^{-1}\of{B_{\varepsilon}^{N}} \text{ offen in }M\\
            \intertext{Außerdem ist $x_0\in V$}
            &\impl \exists\delta > 0\colon B_{\delta}^{M} \subseteq V\\
            \impl &d_N\of{f\of{x}, f\of{x_0}} < \varepsilon\quad\forall x\in B_{\delta}^{M}\of{x_0}\\
            \impl &\forall d_M\of{x, x_0} < \delta\colon d_N\of{f\of{x}, f\of{x_0}} < \varepsilon
        \end{align*}
        (a) $\impl$ (c): Kontraposition. Angenommen es existiert eine Menge $U\subseteq N$, sodass $V\coloneqq f^{-1}\of{U}$ nicht offen ist. Dann existiert ein $x_0\in V$, sodass
        \begin{align*}
            \forall \delta> 0\colon B_{\delta}\of{x_0} &\not\subseteq V
            \intertext{Da $x_0\in V$ ist $f\of{x_0} \in U$. Da $U$ offen ist, gilt}
            \exists \varepsilon_0\colon B_{\varepsilon_0}^{N}\of{f\of{x_0}} &\subseteq U
            \intertext{Sei $\delta_n = \frac{1}{n}$. Da $B_{\delta}\of{x_0}$ keine Teilmenge von $V$ ist, folgt}
            \exists x_n&\in B_{\frac{1}{n}}\of{x_0}\\
            \impl x_n &\not\in V\\
            \impl x_n&\not\in f^{-1}\of{B_{\varepsilon_0}^N\of{f\of{x_0}}}\\
            \impl d_N\of{f\of{x_n}, f\of{x_0}} &> \varepsilon_0\\
            \impl d_M\of{x_n, x_0} &< \frac{1}{n} \text{ aber } d_N\of{f\of{x_N}, f\of{x_0}} > \varepsilon_0
        \end{align*}
        Damit ergibt sich ein Widerspruch zu (a).
    \end{proof}
\end{satz}

\begin{bemerkung}
    Für $N=\R^d$ und $M\subseteq\R$ ist die Stetigkeit von $f: M\fromto \R^d,~x\mapsto\pair{f_1\of{x}, f_2\of{x}, \ldots, f_n\of{x}}$ äquivalent dazu, dass jedes $f_j$ stetig ist.
\end{bemerkung}

\begin{bemerkung}
    \marginnote{[25. Jun]}
    Sei $\pair{M, d}$ ein metrischer Raum. Wenn $g, f: M\fromto\R$ stetig sind, dann sind auch $f+g: M\fromto\R,x\mapsto f\of{x}+g\of{x}$ sowie $f\cdot g: M\fromto\R,~x\mapsto f\of{x}\cdot g\of{x}$ stetig.\\
    Außerdem ist die Menge $D_{g\neq 0} \coloneqq \set{x\in M: g\of{x}\neq 0}$ für ein stetiges $g$ offen. Das heißt wenn $g\of{x_0} \neq 0$, dann ist $D_{g\neq 0}$ eine Umgebung und wir können eine Funktion $\frac{f}{g}: D_{g\neq 0}\fromto\R$ definieren, die stetig in $x_0$ ist.
    \begin{proof}
    (Übung)
    \end{proof}
\end{bemerkung}

\begin{definition}[Homöomorphismus] % Def 9
    Seien $\pair{M, d_M}$ und $\pair{N, d_N}$ metrische Räume. Dann heißt eine Funktion $f: M\fromto N$ Homöomorphismus, wenn sie bijektiv und stetig ist und $f^{-1}: N\fromto M$ auch stetig ist.
\end{definition}

\begin{beispiel}
    \theoremescape
    \begin{enumerate}
        \item Die Funktion
        \begin{align*}
            f: \R&\fromto\pair{-1, 1}\\
            x &\mapsto \frac{x}{1+\abs{x}}
        \end{align*}
        ist ein Homöomorphismus.
        \item Die Funktion
        \begin{align*}
            f: \R^d &\fromto B_{1}\of{0} = \set{x\in\R^d: \norm{x} < 1}\\
            x &\mapsto f\of{x} = \frac{x}{1+\norm{x}}
            \intertext{ist ein Homöomorphismus mit der Umkehrabbildung}
            f^{-1}\of{y} &= \frac{y}{1-\norm{y}}
        \end{align*}
    \end{enumerate}
\end{beispiel}

\begin{definition} % Def 10
    Seien $\pair{M, d_M}$ und $\pair{N, d_N}$ metrische Räume. Eine Funktionenfolge $(f_n)_n$ mit $f_n: M\fromto N$ konvergiert gleichmäßig gegen $f: M\fromto N$, falls
    \begin{align*}
        \forall\varepsilon > 0\ex N\in\N\colon d_N\of{f\of{x}, f_n\of{x}} &< \varepsilon\quad\forall x\in M\fa n\geq N
        \intertext{Das heißt}
        \limsup_{\ntoinf} \sup_{x\in M} d_N\of{f\of{x}, f_n\of{x}} &= 0
    \end{align*}
\end{definition}

\begin{satz}
    Seien $f_n: M\fromto N$ stetig für alle $n\in\N$ und konvergiere $(f_n)_n$ gleichmäßig gegen $f: M\fromto N$. Dann ist $f$ stetig.
    \begin{proof}
        Wir verwenden den $\frac{\varepsilon}{3}$-Trick. Angenommen $\varepsilon > 0$ und $x_0\in M$ beliebig. Wegen der gleichmäßigen Konvergenz gilt
        \begin{align*}
            \exists N\in\N\colon d_N\of{f\of{x}, f_n\of{x}} &< \frac{\varepsilon}{3}\quad\forall n\geq N
            \intertext{$f_N$ ist stetig in $x_0$. Das heißt}
            \exists\delta > 0\colon d_N\of{f_N\of{x}, f_N\of{x_0}} &< \frac{\varepsilon}{3}\quad\forall x\in M, d_M\of{x, x_0} < \delta
        \end{align*}
        Sei nun $x\in B_{\delta}\of{x_0} = \set{y\in M: d_M\of{y, x_0} <\delta}$. Dann folgt durch das Verwenden der Dreiecksungleichung
        \begin{align*}
            d_N\of{f\of{x}, f\of{x_0}} &\leq d_N\of{f\of{x}, f_N\of{x}} + d_N\of{f_N\of{x}, f\of{x_0}}\\
            &\leq d_N\of{f\of{x}, f_N\of{x}} + d_N\of{f_N\of{x}, f_N\of{x_0}} + d_N\of{f_N\of{x_0}, f\of{x_0}}\\
            &<\frac{\varepsilon}{3} + \frac{\varepsilon}{3} + \frac{\varepsilon}{3} = \varepsilon\qedhere
        \end{align*}
    \end{proof}
\end{satz}

\newpage