\subsection{Metrische Räume}
\thispagestyle{pagenumberonly}

\marginnote{[07. Jun]}
Frage: Wie definiert man Abstand auf allgemeinen Mengen?

\begin{beispiel}[Abstände zweier reeller Zahlen]
    Seien $x,y\in\R$. Wir definieren den Abstand über den Betrag. Das heißt $d\of{x,y}\coloneqq \abs{x-y}$.
    Der Abstand hat in diesem Fall die Eigenschaften
    \begin{enumerate}[label=(\roman*)]
        \item $\abs{x-y} \geq 0$ und $\abs{x-y} = 0 \equivalent x=y$
        \item $\abs{x-y} = \abs{y-x}$
        \item $\abs{x-y} \leq \abs{x-z} + \abs{z-y}$ für ein beliebiges $z\in\R$
    \end{enumerate}
\end{beispiel}

\begin{definition}[Metrik]
    Sei $X$ eine Menge und $d: X \times X\fromto \R$ eine Abbildung mit den Eigenschaften
    \begin{enumerate}[label=(\roman*)]
        \item $d\of{x,y} \geq 0~\forall x,y\in X$ und $d\of{x,y} = 0\equivalent x = y$
        \item $d\of{x,y} = d\of{y,x}~\forall x,y\in X$
        \item $d\of{x,y} \leq d\of{x,z} + d\of{z,y}~\forall x,y,z\in X$
    \end{enumerate}
    In diesem Fall nennen wir $d$ eine Metrik auf $X$ und das Paar $\pair{X, d}$ einen metrischen Raum.
\end{definition}

\begin{beispiel}
    \theoremescape
    \begin{enumerate}
        \item Auf $\R$ oder $\C$ definieren wir $d\of{x,y} \coloneqq \abs{x-y}$
        \item Für $x,y\in\R^2$ definieren wir zum Beispiel die euklidische Metrik $d\of{x,y} \coloneqq \sqrt{\pair{x_1 - y_1}^2 + \pair{x_2 - y_2}^2}$
        \item Sei $\pair{X, d}$ ein metrischer Raum und sei $A\subseteq X$. Dann definieren wir die \emph{induzierte Metrik} $d_A: A \times A \fromto \R,~\pair{x,y}\mapsto d\of{x,y}$
        \item Die diskrete Metrik ist definiert durch $d\of{x,y} \coloneqq \begin{cases}
                                                                               0 & x = y\\
                                                                               1 & x\neq y
        \end{cases}$
        \item Die französische Eisenbahnmetrik im $\R^2$ ist definiert durch $d\of{x,y} \coloneqq \sqrt{x_1^2 + x_2^2} + \sqrt{y_1^2 + y_2^2}$ falls $x$ und $y$ nicht auf einer Geraden durch $\pair{0,0}$ liegen und sonst $d\of{x,y} \coloneqq \sqrt{\pair{x_1 - y_1}^2 + \pair{x_2 - y_2}^2}$.
        \item Sei $\pair{X, d}$ ein metrischer Raum. Dann können wir eine Metrik $d_1$ definieren durch
        \begin{align*}
            d_1\of{x,y} &\coloneqq \frac{d\of{x,y}}{1+d\of{x,y}}
            \intertext{Dass $d_1$ die Eigenschaften (i) und (ii) im Sinne der Definition erfüllt, rechnet sich leicht nach. Für (iii) gilt für ein $z\in X$}
            d\of{x,y} &\leq d\of{x,z} + d\of{z,y}
            \intertext{Außerdem gilt für $t\in\R$}
            \frac{t}{1+t} &= 1- \frac{1}{t+1} \tag{monoton steigend}\\
            \impl d_1\of{x,y} = \frac{d\of{x,y}}{1+d\of{x,y}} &\leq \frac{d\of{x,z} + d\of{z,y}}{1+d\of{x,z} + d\of{z,y}}\\
            &= \frac{d\of{x,z}}{1+d\of{x,z} + d\of{z,y}} + \frac{d\of{z,y}}{1+d\of{x,z} + d\of{z,y}}\\
            &\leq \frac{d\of{x,z}}{1+d\of{x,z}} + \frac{d\of{z,y}}{1+d\of{z,y}} = d_1\of{x,z} + d_1\of{z,y}
        \end{align*}
    \end{enumerate}
\end{beispiel}

\subsection{Normierte Vektorräume}

\begin{definition}[Vektorraum]
    Ein Vektorraum $V$ über $\K\in\set{\R, \C}$ ist eine Menge auf der es eine Vektoraddition
    \begin{align*}
        +: V \times V&\fromto V\\
        \pair{x,y}&\mapsto x+y
        \intertext{und eine (skalare) Multiplikation}
        \cdot: \K \times V&\fromto V\\
        \pair{\alpha, x}&\mapsto \alpha \cdot x
    \end{align*}
    gibt, welche den üblichen Axiomen aus der Algebra\footnote{Zum Beispiel $x+y = y+x$, $\alpha\pair{x+y} = \alpha x + \alpha y$, etc. Für eine genaue Aufzählung siehe \textsc{Lineare Algebra I}} genügen.
\end{definition}

\begin{definition}[Norm]
    Eine Norm auf einem VR $V$ ist eine Abbildung $\norm{\cdot}: V \fromto \R,~x\mapsto\norm{x}$ mit den Eigenschaften
    \begin{enumerate}[label=(\roman*)]
        \item $\norm{x} \geq 0$ und $\norm{x} = 0 \equivalent x = 0~\forall x\in V$
        \item $\norm{\lambda x} = \abs{\lambda}\norm{x}~\forall \lambda\in\K, x\in V$
        \item $\norm{x+y} \leq \norm{x} + \norm{y}~\forall x,y\in V$
    \end{enumerate}
\end{definition}

\begin{definition}[Normierter Vektorraum]
    Ein normierter Vektorraum ist ein Paar $\pair{V, \norm{\cdot}}$ aus einem VR $V$ und einer Norm $\norm{\cdot}$ auf $V$.
\end{definition}

\begin{satz} % Satz 3
    Ist $\pair{V, \norm{\cdot}}$ ein normierter VR, so wird durch $d\of{x,y}\coloneqq \norm{x-y}$ eine Metrik auf $V$ definiert.
    \begin{proof}
        Die Axiome der Metrik folgen unmittelbar aus den Axiomen der Norm.
    \end{proof}
\end{satz}

\begin{bemerkung}[Halbnorm]
    Gelten für eine Abbildung die Norm-Eigenschaften (ii) und (ii), aber statt (i) nur $\norm{x} \geq 0~\forall x\in V$. Dann heißt $\norm{\cdot}$ eine Halbnorm. Ein Beispiel dafür im $\R^2$ wäre die Abbildung $\pair{x_1~x_2}\mapsto x_1$.
\end{bemerkung}

\begin{beispiel}
    \theoremescape
    \begin{enumerate}
        \item Sei $V$ ein euklidischer reeller VR mit symmetrischem, positiv-definitem Skalarprodukt $\sprod{\cdot,\cdot}$. Dann ist $\norm{x}\coloneqq \sqrt{\sprod{x,x}}$ eine Norm.
        \item Analog funktioniert der Fall, dass $V$ ein komplexer VR mit positiv-definiter sesquilinearer Bilinearformen $\sprod{\cdot,\cdot}$ ist. Das heißt $\sprod{\cdot, \cdot}: V\times V\fromto\C$ hat die Eigenschaften
        \begin{enumerate}[label=(\roman*)]
            \item $\sprod{x,y} = \conj{\sprod{y,x}}$ und $\sprod{x,x} \geq 0~\forall x,y\in V$ (\footnote{Die Forderung $\sprod{x,x} \geq 0$ ist wohldefiniert, weil durch die anderen Eigenschaften folgt, dass $\forall x\in V\colon \sprod{x,x}\in\R$})
            \item $\sprod{x, u+w} = \sprod{x,u} + \sprod{x,w}$ sowie $\sprod{u+w, y} = \sprod{u, y} + \sprod{w, y}$
            \item $\sprod{x, \alpha y} = \alpha\sprod{x,y}$ sowie $\sprod{\alpha x, y} = \conj{\alpha}\sprod{x,y}$ (\footnote{Physiker-Konvention, anders herum als in der Vorlesung \textsc{Lineare Algebra I/II}})
        \end{enumerate}
        Dann definieren wir eine Norm durch $\norm{x} \coloneqq \sqrt{\sprod{x,x}}$.
        \item Im $\R^d$ definieren wir für zwei Vektoren $x=\pair{x_1, \ldots, x_d}$ und $y=\pair{y_1, \ldots, y_d}$ ein Skalarprodukt durch
        \begin{align*}
            \sprod{x,y} &\coloneqq \sum_{j=1}^{d} x_j \cdot y_j
            \intertext{Und eine Norm durch}
            \norm{x} &\coloneqq \norm{x}_2 \coloneqq \sqrt{\sprod{x,x}} = \sqrt{\sum_{j=1}^{d} \abs{x_j}^2}
        \end{align*}
        \item Wir definieren die Maximums-Norm auf $\R^d$ durch $\norm{x}_{\infty} \coloneqq \max_{1\leq j\leq d} \abs{x_j}$. Daraus folgt auch $\norm{x}_{\infty} \leq \norm{x}_2 \leq \sqrt{d}\cdot\norm{x}_{\infty}$.
        \item Wir definieren die Manhattan-Norm auf $\R^d$ durch
        \begin{align*}
            \norm{x}_1 &\coloneqq \sum_{j=1}^{d} \abs{x_j}
        \end{align*}
        und es gilt $\norm{x}_{\infty} \leq \norm{x}_1 \leq d \cdot \norm{x}_{\infty}$.
        \item Sei $X$ eine beliebige Menge und $\mathcal{B}\of{X}$ der Vektorraum der reellwertigen beschränkten Funktionen $f: X\fromto \R$. Dann definieren wir
        \begin{align*}
            \norm{f}_{L^{\infty}\of{x}} &\coloneqq \sup\set{\abs{f\of{x}}: x\in X} = \sup_{x\in X} \abs{f\of{x}}
        \end{align*}
        als eine Norm auf $\mathcal{B}\of{X}$. Das ist eine Verallgemeinerung der Maximumsnorm.
    \end{enumerate}
\end{beispiel}

\subsection{[*] Umgebungen und offene Mengen}

\begin{definition}[Umgebung]
    Sei $\pair{X, d}$ ein metrischer Raum
    \begin{enumerate}[label=(\alph*)]
        \item Für $a\in X$, $r > 0$ heißt $B_r\of{a} \coloneqq \set{x\in X ~\middle\vert~ d\of{a, x} < r}$ die (offene) Kugel um $a$ mit dem Radius $r$ und dem Mittelpunkt $a$.
        \item Eine Teilmenge $U\subseteq X$ heißt \emph{Umgebung} von $x\in X$, falls $\exists\varepsilon > 0$, sodass $B_{\varepsilon}\of{x} \subseteq U$. Insbesondere ist $B_{\varepsilon}\of{x}$ eine Umgebung von $x$. Wir nennen $B_{\varepsilon}\of{x}$ die $\varepsilon$-Umgebung von $x$.
    \end{enumerate}
\end{definition}

\begin{satz}[Hausdorffsches Trennungsaxiom in metrischen Räumen]
    \label{satz:hausdorff-trennungsaxiom}
    Sei $\pair{X, d}$ ein metrischer Raum. Dann existieren zu $x,y\in X$ mit $x\neq y$ $\varepsilon$-Umgebungen mit $B_{\varepsilon}\of{x} \cap B_{\varepsilon}\of{y} = \emptyset$.
    \begin{proof}
        Wir wählen $\varepsilon = \frac{1}{2}\cdot d\of{x,y} > 0$. Angenommen $\exists z\in B_{\varepsilon}\of{x} \cap B_{\varepsilon}\of{y}$. Dann gilt
        \begin{align*}
            2\varepsilon &= d\of{x,y} \leq d\of{x,z} + d\of{z,y}\\
            &< \varepsilon + \varepsilon = 2\varepsilon\qedhere
        \end{align*}
    \end{proof}
\end{satz}

\begin{definition}[Offene Menge]
    \marginnote{[11. Jun]}
    Sei $\pair{X, d}$ ein metrischer Raum. Eine Teilmenge $U\subseteq X$ heißt offen, wenn $\forall x\in U\ex \varepsilon > 0\colon B_{\varepsilon}\of{x}\subseteq U$.
\end{definition}

\begin{beispiel}
    Seien $a, b\in\R$ und $a < b$
    \begin{enumerate}
        \item Dann sind $\pair{a, b}$, $\pair{a, \infty}$ und $\pair{-\infty, a}$ offen
        \item Die Mengen $\linterv{a,b}$, $\interv{a,b}$, $\linterv{a,\infty}$ sind nicht offen, weil für den ersten Fall $\pair{a-\varepsilon, a + \varepsilon}\nsubseteq \linterv{a,b}$.
        \item In jedem metrischen Raum $\pair{X, d}$ ist für beliebige $a\in X$, $r > 0$ die Menge $B_r\of{a}$ offen. Deshalb heißt $B_r\of{a}$ die offene Kugel um $a$ mit Radius $r$.
        \item In $\R^d$ gilt $U\subseteq\R^d$ ist offen bezüglich $\norm{\cdot}_{\infty} \equivalent U$ ist offen bezüglich $\norm{\cdot}_2$
    \end{enumerate}
    \begin{proof}[Beweis für 3.]
        Sei $x\in B_{r}\of{a}$ und $\varepsilon \coloneqq r - d\of{a, x} > 0$. Sei $z\in B_{\varepsilon}\of{x}$. Dann gilt
        \begin{align*}
            d\of{a,z} \leq d\of{a,x} + d\of{x,z} &< d\of{a, x} + r - d\of{a,x} = r\\
            \impl d\of{a,z} &< r\\
            \impl z\in &B_r\of{a}\qedhere
        \end{align*}
    \end{proof}
    \begin{proof}[Beweis für 4.]
        Sei $B_{\varepsilon}^{\infty}\of{x}\coloneqq\set{y\in \R^d: \norm{x-y}_{\infty} < \varepsilon}$ und $B_{\varepsilon}^{(2)}\of{x} \coloneqq \set{y\in\R^d: \norm{x-y}_2 < \varepsilon}$. Für ein $y\in B_{\frac{\varepsilon}{\sqrt{d}}}^{(2)}\of{x}$ gilt
        \begin{align*}
            \sqrt{ \sum_{i=1}^{d} \pair{y_i - x_i}^2} &< \frac{\varepsilon}{\sqrt{d}}\\
            \impl \sum_{i=1}^{d} \pair{y_i - x_i}^2 &< \frac{\varepsilon^2}{d}\\
            \impl \max_{1 \leq i \leq d} \abs{x_i - y_i} &< \varepsilon\\
            \impl y&\in B_{\varepsilon}^{\infty}\of{x}
            \intertext{Außerdem lässt sich zeigen, dass daraus auch folgt, dass $y\in B_{\varepsilon}^{(2)}\of{x}$. Das heißt}
            B_{\frac{\varepsilon}{\sqrt{d}}}^{(2)}\of{x} \subseteq B_{\varepsilon}^{\infty}\of{x} &\subseteq B_{\varepsilon}^{(2)}\of{x}\qedhere
        \end{align*}
    \end{proof}
\end{beispiel}

\begin{satz} % Satz 7
    \label{satz:offene-mengen-metr}
    Für die offenen Mengen eines metrischen Raums $\pair{X, d}$ gilt
    \begin{enumerate}[label=(\roman*)]
        \item $\emptyset$ und $X$ sind offen
        \item Sind $U$ und $V$ offen, so ist auch $U\cap V$ offen
        \item Ist $I$ eine beliebige Indexmenge und $(U_j)_{j\in I}$ eine Familie offener Teilmengen von $X$. So ist $\bigcup_{j\in I} U_j$ offen.
    \end{enumerate}

    \begin{proof}[Beweis von (ii)]
        Sei $x\in U\cap V$. Da $U$ und $V$ offen sind, gibt es $\varepsilon_1 > 0$ und $\varepsilon_2 > 0$ mit $B_{\varepsilon_1}\of{x}\subseteq U$, $B_{\varepsilon_2}\of{x}\subseteq V$. Sei $\varepsilon\coloneqq \min\set{\varepsilon_1, \varepsilon_2} \impl B_{\varepsilon}\of{x}\subseteq U\land B_{\varepsilon}\of{x}\subseteq V$. Damit gilt $B_{\varepsilon}\of{x}\subseteq U\cap V$.
    \end{proof}

    \begin{proof}[Beweis von (iii)]
        Sei $x\in \bigcup_{j\in I} U_j \impl \exists i\colon x\in U_{i}$. Außerdem ist $U_{i}$ offen. Das heißt es existiert ein $\varepsilon > 0$, sodass $B_{\varepsilon}\of{x}\subseteq U_{i} \impl B_{\varepsilon}\of{x}\subseteq \bigcup_{j\in I} U_{j}$.
    \end{proof}
\end{satz}

\begin{bemerkung}
    Seien $U_1, \ldots, U_k$ offen. Dann folgt aus Satz~\ref{satz:offene-mengen-metr}, dass $U_1\cap U_2\cap\dots\cap U_k$ offen ist. Das gilt allerdings nur für $k < \infty$.\\
    Wir betrachten für einen Schnitt über unendlich viele Mengen das folgende Beispiel:
\end{bemerkung}

\begin{beispiel}[Schnitt über unendlich viele offene Mengen]
    Sei $U_n = \pair{-\frac{1}{n}, 1+\frac{1}{n}}$. Dann ist $U_n$ offen $\forall n\in\N$. Allerdings ist $\bigcap_{n=1}^{\infty} U_n = \interv{0, 1}$ nicht offen.
\end{beispiel}

\begin{definition}[Abgeschlossene Menge]
    In einem metrischen Raum $\pair{X, d}$ nennen wir eine Menge $A\subseteq X$ abgeschlossen, wenn ihr Komplement $A^{\mathrm{C}}\coloneqq X \exclude A = \set{y\in X: y\not\in A}$ offen ist.
\end{definition}

\begin{bemerkung}
    Eine andere (aber äquivalente) Definition von Abgeschlossenheit wurde 1884 von Cantor gegeben. Diese Definition basiert auf Folgen.
\end{bemerkung}

\begin{definition}[Konvergenz in metrischen Räumen]
    Sei $\pair{X, d}$ ein metrischer Raum. Eine Folge $(x_n)_n\subseteq X$ konvergiert gegen den Punkt $a\in X$, wenn
    \begin{align*}
        &\forall\varepsilon>0\ex N\in\N\colon d\of{a, x_n} < \varepsilon\quad\forall n> N
        \intertext{das heißt}
        \equivalent &\forall\varepsilon > 0\ex N\in\N\colon x_n\in B_{\varepsilon}\of{a}\quad\forall n> N\\
        \equivalent &\forall\varepsilon > 0 \text{ ist } x_n\in B_{\varepsilon}\of{a} \text{ für fast alle $n$}
    \end{align*}
    Wir schreiben dann $\biglim{\ntoinf} x_n = a$.
\end{definition}

\begin{definition}[Folgenabgeschlossenheit]
    Sei $\pair{X, d}$ ein metrischer Raum. Eine Menge $A\subseteq X$ ist folgenabgeschlossen, falls für jede Folge $(x_n)_n \subseteq A$, die gegen einen Punkt $a\in X$ konvergiert, auch gilt $a\in A$.
\end{definition}

\begin{beispiel}
    Wir betrachten die Menge $A = \rinterv{0, 1}$ und die Folge $x_n = \frac{1}{n}$. Dann liegt $\lim_{\ntoinf} x_n$ nicht in $A$. Das heißt $A$ ist nicht folgenabgeschlossen. Die Menge $\interv{0, 1}$ hingegen schon.
\end{beispiel}

\begin{satz} % Satz 11
    \label{satz:vergleich-folgen-abgeschlossen}
    Sei $\pair{X, d}$ ein metrischer Raum. Für $A\subseteq X$ gilt, $A$ ist genau dann abgeschlossen, wenn $A$ folgenabgeschlossen ist.
\end{satz}

Um diesen Satz zu beweisen, benötigen wir zunächst noch folgende Lemmata

\begin{lemma} % Lemma 12
    \label{lemma:komplement-folgen-abgeschlossen}
    Sei $\pair{X, d}$ ein metrischer Raum. Ist $U\subseteq X$ offen, so ist $U^{\mathrm{C}}$ folgenabgeschlossen.

    \begin{proof}
        Angenommen es gibt eine Folge $(x_n)_n \subseteq U^{\mathrm{C}}$ mit $x_n\fromto a$, aber $a\not\in U^{\mathrm{C}}$. Das heißt $a\in U$. Da $U$ offen ist, existiert eine Kugel $B_{\varepsilon}\of{a} \subseteq U$ (für ein $\varepsilon > 0$). Da $x_n\fromto a$ für $\ntoinf$ gilt
        \begin{align*}
            \exists N\in\N\colon &d\of{a, x_n} < \varepsilon\quad\forall n> N \\
            \equivalent \exists N\in\N\colon &x_n \in B_{\varepsilon}\of{a}\quad\forall n > N\\
            \impl &x_n\not\in U^{\mathrm{C}}\quad\forall n > N\tag*{(Widerspruch)\qedhere}
        \end{align*}
    \end{proof}
\end{lemma}

\begin{lemma}
    \label{lemma:komplement-offen}
    Sei $\pair{X, d}$ ein metrischer Raum und $A\subseteq X$ folgenabgeschlossen. Dann ist $A^{\mathrm{C}}$ offen.

    \begin{proof}
        Angenommen $A^{\mathrm{C}}$ ist nicht offen. Das heißt $\exists x_0 \in A^{\mathrm{C}}$, sodass für jedes $\varepsilon > 0$ die Kugel $B_{\varepsilon}\of{x_0}$ nicht ganz in $A^{\mathrm{C}}$ enthalten ist. Das heißt
        \begin{align*}
            \forall\varepsilon > 0\colon B_{\varepsilon}\of{x_0}\cap A \neq \emptyset
        \end{align*}
        Wir wählen eine Folge $(x_n)_n\subseteq A$ mit $x_n\in B_{\frac{1}{n}}\of{x_0}$. Also gilt $x_n\fromto x_0\in A^{\mathrm{C}}$. Das heißt $A$ ist nicht folgenabgeschlossen. Widerspruch.
    \end{proof}
\end{lemma}

\begin{proof}[Beweis von Satz~\ref{satz:vergleich-folgen-abgeschlossen}]
    ~\\
    \anf{$\impl$} Wenn $A$ abgeschlossen ist, dann gilt nach Definition, dass $A^{\mathrm{C}}$ offen ist. Mit Lemma~\ref{lemma:komplement-folgen-abgeschlossen} folgt dann, dass $A$ folgenabgeschlossen ist.\\
    \anf{$\Leftarrow$} Sei $A$ folgenabgeschlossen. Dann folgt direkt aus Lemma~\ref{lemma:komplement-offen}, dass $A$ abgeschlossen.
\end{proof}

\newpage

\begin{beispiel}
    \marginnote{[14. Jun]}
    Seien $A_1\subseteq\R^k$, $A_2\sbset\R^m$ abgeschlossen. Dann ist auch $A_1\times A_2 \subseteq \R^k \times \R^m$ abgeschlossen.
    \begin{proof}
        Sei $\pair{x,y}\in \pair{A_1 \times A_2}^{\mathrm{C}} \supseteq A_1^C \times \R^m \cup \R^k \times A_2^C$.\\[.5\baselineskip]
        \textsc{Fall 1}: $x \in A_1^C$. Dann gilt wegen der Abgeschlossenheit von $A_1$
        \begin{align*}
            \exists \varepsilon > 0\colon B_{\varepsilon}\of{x} &\subseteq A_1^C\\
            \impl B_{\varepsilon}\of{x, y} &\subseteq A_1^{\mathrm{C}} \times \R^m
            \intertext{Aber $B_{\varepsilon}\of{x,y}$ ist eine offene Menge mit}
            B_{\varepsilon}\of{x,y} \subseteq A^{\mathrm{C}}_1 \times \R^m &\subseteq \pair{A_1 \times A_2}^{\mathrm{C}}
            \intertext{\textsc{Fall 2}: $y\in A_2^{\mathrm{C}}$. Dann gilt analog zu \textsc{Fall 1}}
            \impl \exists \varepsilon > 0\colon B_{\varepsilon}\of{x,y} &\subseteq \pair{A_1 \times A_2}^{\mathrm{C}}
        \end{align*}
        Das heißt $\pair{A_1 \times A_2}^{\mathrm{C}}$ ist offen und damit ist $\pair{A_1 \times A_2}$ abgeschlossen.
    \end{proof}
\end{beispiel}

\begin{bemerkung}
    \theoremescape
    \begin{enumerate}[label=(\roman*)]
        \item In jedem metrischen Raum $\pair{X, d}$ sind die Mengen $\emptyset$ und $X$ sowohl offen als auch abgeschlossen
        \item Das Intervall $\linterv{a,b}\subseteq\R$ ist nicht offen und nicht abgeschlossen
    \end{enumerate}
\end{bemerkung}

\subsection{Grundzüge der Topologie}

\begin{definition}[Topologie]
    Sei $X$ eine Menge. Dann heißt ein Mengensystem $\Tau\sbset\mathcal{P}\of{X}$ eine Topologie auf $X$, falls
    \begin{enumerate}[label=(\roman*)]
        \item $\emptyset, X\in\Tau$
        \item $U, V\in \Tau \impl U \cap V \in \Tau$ und
        \item Für eine beliebige Indexmenge $I$ mit $\forall j\in I\colon V_j \in \Tau$ folgt $\displaystyle\bigcup_{j\in I} V_j \in \Tau$
    \end{enumerate}
    Ein topologischer Raum ist ein Paar $\pair{X, \Tau}$. $V\subseteq X$ heißt offen, falls $V\in\Tau$ und $A\subseteq X$ heißt abgeschlossen, falls $A^{\mathrm{C}} \in\Tau$.
\end{definition}

\begin{beispiel}
    \theoremescape
    \begin{enumerate}
        \item Das System von offenen Mengen eines metrischen Raums $X$ ist eine Topologie
        \item $\R^d$ ist ein topologischer Raum (mit dem System der offenen Mengen als implizierte Topologie)
        \item Auf jeder Menge $X$ gibt es mindestens 2 Topologien: $\Tau_0 \coloneqq \set{\emptyset, X}$ und die feinste Topologie $\Tau_1 = \mathcal{P}\of{X}$. Ist $\abs{X} \geq 2$, so ist $\Tau_0 \neq \Tau_1$
        \item Sei $\pair{X, \Tau}$ ein topologischer Raum und $Y\subseteq X$ eine Teilmenge von $X$. Wir definieren ähnlich zur induzierten Metrik eine \emph{induzierte Topologie} (Relativ-Topologie)
        \begin{align*}
            \Tau_{Y} \coloneqq \Tau \cap Y \coloneqq \set{U \cap Y \colon U\in \Tau}
        \end{align*}
        Dann ist $\pair{Y, \Tau_Y}$ ein topologischer Raum.
    \end{enumerate}
\end{beispiel}

\begin{bemerkung}
    Ist $Y$ nicht offen in $X$, so ist $V\in \Tau_Y$ nicht notwendigerweise offen in $X$.
\end{bemerkung}

\begin{definition}[Umgebung]
    Sei $\pair{X, \Tau}$ ein topologischer Raum und $x\in X$ (Punkt in $X$). Eine Teilmenge $V\subseteq X$ heißt Umgebung von $x$, falls es eine Menge $U\in\Tau$ gibt mit $x\in U \subseteq V$.
\end{definition}

\begin{satz} % Satz 16
    Sei $\pair{X, \Tau}$ ein topologischer Raum. Eine Menge $V\subseteq X$ ist genau dann offen, wenn $V$ eine Umgebung für jeden Punkt $x\in V$ ist.
    \begin{proof}
        \anf{$\impl$} Sei $V$ offen. Dann existiert für jedes $x\in V$ eine offene Menge $U\in\Tau$ mit $x\in U\subseteq V$. Also ist $V$ eine Umgebung jedes Punktes $x\in V$.\\[.5\baselineskip]
        \anf{$\Leftarrow$} Sei $V$ eine Umgebung für alle Punkte $x\in V$. Dann gilt $\forall x\in V\ex U_x \in \Tau\colon x\in U_x \subseteq V$. Dann ist $V=\bigcup_{x\in V} U_x$. Das heißt $V$ ist als Vereinigung offener Mengen offen.
    \end{proof}
\end{satz}

\begin{definition}[Hausdorff-Raum]
    Ein topologischer Raum heißt Hausdorff-Raum, falls das Hausdorffsche Trennungsaxiom gilt. Das heißt zu zwei Punkten $x,y\in X$ mit $x\neq y$ existieren offene Mengen $U,V\in\Tau$ mit $x\in U$, $y\in V$ und $U\cap V = \emptyset$.
\end{definition}

\begin{beispiel}
    \theoremescape
    \begin{enumerate}
        \item Nach Satz~\ref{satz:hausdorff-trennungsaxiom} ist jeder metrische Raum ein Hausdorff-Raum.
        \item Sei $X=\set{0, 1}$ und $\Tau\coloneqq \set{\emptyset, \set{0}, \set{0, 1}}$ eine Topologie. Dann ist $\pair{X, \Tau}$ kein Hausdorff-Raum. Um das einzusehen, betrachten wir $x=0$ und $y=1$. Für diese zwei Elemente finden wir keine entsprechenden Mengen.
    \end{enumerate}
\end{beispiel}

\subsection{[*] Berührpunkt, Häufungspunkt und Randpunkt}

\begin{mdframed}
    \begin{center}
        Für die folgenden Definitionen sei $\pair{X, \Tau}$ ein topologischer Raum und $A\subseteq X$.
    \end{center}
\end{mdframed}

\begin{definition}[Berührpunkt]
    Ein Punkt $x\in X$ heißt Berührpunkt von $A$, wenn in jeder offenen Umgebung $U$ von $x$ mindestens ein Punkt aus $A$ liegt. Das heißt $U\cap A \neq \emptyset$ für alle offenen Mengen $U$ mit $x\in U$.
\end{definition}

\begin{definition}[Häufungspunkt]
    Ein Punkt $x\in X$ heißt Häufungspunkt von $A$, wenn für jede offene Umgebung $U$ von $x$ ein von $x$ verschiedener Punkt in $A$ liegt. Das heißt $A\cap \pair{U\exclude\set{x}} \neq \emptyset$ für alle offenen Mengen $U$ mit $x\in U$.
\end{definition}

\begin{definition}[Randpunkt]
    Ein Punkt $x\in X$ heißt Randpunkt von $A$, falls es in jeder offenen Umgebung $U$ von $x$ mindestens einen Punkt aus $A$ und einen Punkt aus $A^{\mathrm{C}}$ gibt. Das heißt für alle offenen Mengen $U$ mit $x\in U$ ist $U\cap A \neq \emptyset$ und $U\cap A^{\mathrm{C}} \neq \emptyset$. Wir schreiben $\partial A$ für die Menge aller Randpunkte von $A$.
\end{definition}

\begin{satz} % Satz 19
    \marginnote{[18. Jun]}
    \label{satz:top-rand}
    Ist $\pair{X, \Tau}$ ein topologischer Raum und $A\subseteq X$. Dann gilt
    \begin{enumerate}[label=(\roman*)]
        \item $A \exclude \partial A$ ist offen
        \item $A \cup \partial A$ ist abgeschlossen
        \item $\partial A$ ist abgeschlossen
    \end{enumerate}
    \begin{proof}
        \theoremescape
        \begin{enumerate}[label=(\roman*)]
            \item Sei $x\in A \exclude \partial A$ beliebig. Dann folgt es existiert eine offene Umgebung $V$ von $x$ mit
            \begin{align*}
                V \cap A^{\mathrm{C}} &= V\cap \pair{X\exclude A} = \emptyset\tag{(1)}
                \intertext{denn ansonsten wäre $x\in\partial A$}
                \impl V \cap \partial A &= \emptyset\tag{(2)}
                \intertext{Denn wäre $V\cap \partial A \neq \emptyset$. Dann würde folgen}
                \impl \exists y&\in V \cap \partial A\\
                \impl V \text{ offene Umgebung von  } y&\in \partial A\\
                \impl V\cap A^{\mathrm{C}} &\neq \emptyset\tag{Widerspruch zu (1)}
                \intertext{Mit (1) und (2) folgt}
                V \cap\pair{\partial A \cup A^{\mathrm{C}}} &= \emptyset\\
                \equivalent V \subseteq A \exclude \partial A\\
                \impl A\exclude \partial A \text{ ist offen }
            \end{align*}
            \item $A^{\mathrm{C}} = X \exclude A$. Aus der Definition des Randes folgt $\partial A = \partial A^{\mathrm{C}}$. Nach (i) gilt dann
            \begin{align*}
                A^{\mathrm{C}} \exclude \partial A = A^{\mathrm{C}} \exclude \partial A^{\mathrm{C}} \text{ ist offen }\\
                \equivalent \pair{A^{\mathrm{C}} \exclude \partial A}^{\mathrm{C}} \text{ ist abgeschlossen }\\
                \equivalent \pair{A^{\mathrm{C}} \exclude \partial A}^{\mathrm{C}} &= X\exclude\pair{A^{\mathrm{C}} \exclude \partial A}\\
                &= X\exclude A^{\mathrm{C}} \cup \partial A = A\cup \partial A \text{ ist abgeschlossen }
            \end{align*}
            \item
            \begin{align*}
                \partial A &= \pair{A\cup \partial A} \cap \pair{A^C \cup \partial A}\\
                &= \pair{A \cup \partial A} \cap \pair{A^C \cup \partial A^{\mathrm{C}}} \text{ ist abgeschlossen}
            \end{align*}
        \end{enumerate}
    \end{proof}
\end{satz}

\begin{definition}[Inneres und Abschluss, abgeschlossene Hülle]
    Sei $\pair{X, \Tau}$ ein topologischer Raum und $A\subseteq X$.
    \begin{enumerate}[label=(\roman*)]
        \item $U\of{A} \coloneqq \set{U\in\Tau: U\subseteq A}= $ alle offenen Teilmengen von $A$.\\
        Das Innere von $A = A^{\circ} \coloneqq \bigcup_{\sigma\in U\of{A}} \sigma= $Vereinigung aller offenen Teilmengen von $A$
        \item $B\of{A} \coloneqq \set{B\subseteq X \text{ abgeschlossen}: A \subseteq B} \neq \emptyset$ (da $X\in B\of{A}$)\\
        Der Abschluss von $B$ sei $\overline{B} =$ abgeschlossene Hülle von $B \coloneqq \bigcap_{B\in B\of{A}} B$ abgeschlossen
    \end{enumerate}
    Damit gilt $A \subseteq \overline{A}$.
\end{definition}

\begin{bemerkung}
    $A^{\circ} =$ größte offene Teilmenge von $A$ und $\overline{A} = $ kleinste abgeschlossene Menge, die $A$ enthält.
\end{bemerkung}

\begin{satz} % Satz 21
    $A^{\circ} = A \exclude \partial A$ und $\overline{A} = A \cup \partial A$.
    \begin{proof}
        \textsc{Teil 1}: Nach Satz~\ref{satz:top-rand} ist $A\exclude \partial A$ offen und $A \exclude \partial A \subseteq A$. Damit folgt $A \exclude \partial A \subseteq A^{\circ}$.\\
        Damit ist noch zu zeigen, dass $A^{\circ} \subseteq A \exclude \partial A$. Ist $U\subseteq A$ offen
        \begin{align*}
            \impl U\cap A^{\mathrm{C}} &= \emptyset \\
            \impl U \cap \partial A &= \emptyset
        \end{align*}
        Falls nicht $\exists y \in U \cap \partial A$, dann folgt $U\cap A^{\mathrm{C}} \neq \emptyset$. Das ergibt einen Widerspruch.\\
        Das heißt für jede offene Teilmenge $U\subseteq A$ gilt $U\subseteq A\exclude \partial A$.
        \begin{align*}
            A = \bigcup_{U\in U\of{A}} U \subseteq A \exclude \partial A\\
            \impl A^{\circ} \subseteq A \exclude \partial A\\
            \impl A \exclude \partial A = A^{\circ}
        \end{align*}
        \textsc{Teil 2}: Behauptung: Aus $B\subseteq X$ ist abgeschlossen und $A \subseteq B$ folgt $\partial A \subseteq B$. Angenommen die Behauptung ist falsch. Dann würde gelten $B^{\mathrm{C}} \cap \partial A \neq \emptyset$.\\
        $x\in \partial A \cap B^{\mathrm{C}}$. $B^{\mathrm{C}}$ ist offene Umgebung von $x$. Nach Definition von $\partial A$ ist $A\cap B^{\mathrm{C}}\neq \emptyset$. Das heißt $A \subseteq B$. (Widerspruch).\\
        Aus der Behauptung folgt jetzt also $B \supseteq A \cup \partial A~\forall B\supseteq A$ und $B$ ist abgeschlossen
        \begin{align*}
            \impl A \cup \partial A \subseteq \bigcap_{B\in B\of{A}} B = \overline{A}
        \end{align*}
        Andererseits ist nach Satz~\ref{satz:top-rand} $A \cup \partial A$ abgeschlossen und sicherlich $A \subseteq A \cup \partial A$. Daraus folgt $\overline{A} \subseteq A \cup\partial A$. Damit folgt $\overline{A} = A \cup \partial A$.
    \end{proof}
\end{satz}

\newpage